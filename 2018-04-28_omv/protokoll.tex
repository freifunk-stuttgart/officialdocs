\documentclass[a4paper]{scrartcl}
\usepackage[utf8]{inputenc}
\usepackage{caption}
\usepackage{pdfpages}
\usepackage[ngerman]{babel}
\usepackage{url}
\usepackage{ltxtable}
\usepackage{booktabs}
\usepackage{textcomp}
\usepackage{float}
\parindent0pt
\date{28. April 2018}

\title{Ordentliche Mitgliederversammlung des Freifunk Stuttgart e.V.}

\begin{document}

\maketitle

\tableofcontents

\clearpage

\listoftables

\clearpage

\section{Begrüßung}
Philippe Käufer eröffnet die Versammlung um 09:52 Uhr.

\section{Feststellung ordnungsgemäße Einberufung, Beschlussfähigkeit, Tagesordnung}
\subsection{Ladung}
Es wird festgestellt, dass ordnungsgemäß und rechtzeitig geladen wurde. Alle Anwesenden bestätigen, eine fernschriftliche Einladung erhalten zu haben.

\subsection{Feststellung der Beschlussfähigkeit}
\begin{itemize}
\item es sind 14 Personen anwesend.
\item es sind 14 von 54 stimmberechtigten Vereinsmitgliedern anwesend.
\item 15\% der Vereinsmitglieder sind für die Beschlussfähigkeit nötig, die Bedingung ist folglich erfüllt.
\end{itemize}
\subsection{Gäste}
Die Vereinsmitglieder werden befragt, ob Gäste beim Treffen zugelassen sein sollen. Darüber wird per Handzeichen abgestimmt. Die Zulassung von Gästen wird einstimmig beschlossen.

\subsection{Tagesordnung}
Die vorläufige Tagesordnung wird einstimmig angenommen.

Ferner gilt die Geschäftsordnung laut github-Repository\footnote{\url{https://github.com/freifunk-stuttgart/officialdocs/blob/master/go_gv/go_gv.pdf}} fort.

\section{Wahl der Versammlungsämter}
\begin{itemize}
\item Für das Amt des Versammlungsleiters kandidiert Philippe Käufer. Der Vorschlag wird einstimmig angenommen, der Gewählte nimmt das Amt an.
\item Für das Amt des Protokollanten kandidiert Thomas Renger. Der Vorschlag wird einstimmig angenommen, der Gewählte nimmt das Amt an.
\item Für das Amt des Wahlleiters kandidiert Leonard Penzer. Der Vorschlag wird einstimmig angenommen, der Gewählte nimmt das Amt an.
\item Laura Hageneder wird durch den Wahlleiter als Wahlhelferin bestellt. Sie nimmt das Amt an.
\end{itemize}

\section{Tätigkeitsberichte Vorstand und Kassenprüfer, Entlastung}
\subsection{Vorstandsbericht}
Es berichtet Philippe Käufer.

\subsubsection{Community und Veranstaltungen}
\begin{itemize}
\item Subcommunities (Beuren, Göppingen, Esslingen, Frickenhausen)
\item Umsonst und Draußen
\item Verleih der Event-Hardware (im vergangenen Jahr nicht so oft)
\item FFBW:camp
\item Freifunkseminar im Evangelischen Medienhaus
\end{itemize}

\subsubsection{Politik}

\begin{itemize}
\item Digitales Ehrenamt jetzt
\item Politiker Briefaktion
\item Vernetzung mit der Bundes-, Landes- und Kommunalpolitik
\item Zusammenarbeit mit den Gemeinden Esslingen und Frickenhausen
\end{itemize}

\subsubsection{Verein}

\begin{itemize}
\item Haftpflichtversicherung
\item Anmeldung der Providertätigkeit bei der BNetzA
\item Neues Gateway (von einem Göppinger Freifunker gestiftet und von der Gateway-Admin-Community betrieben)
\item Teilnahme des Vereins Freie Netze an Public Money? Public Code!
\item Pressearbeit
\item Vorstellung des FFS im Freifunk-Radio
\item Abschaffung der Störerhaftung (zum dritten mal)
\end{itemize}

Insgesamt wurden sieben Vorstandssitzungen abgehalten.

\subsection{Bericht des Schatzmeisters}
Es berichtet Christoph Altrock. Die wichtigsten Eckdaten:
\begin{itemize}
\item Gesamt-Ertrag: 4203,19 \texteuro
\item Gesamt-Aufwand: 1938,06 \texteuro
\item Netto Ertrag: 2265,13 \texteuro
\item Summe Bestandskonten: 837,35 \texteuro (Kasse) + 5603,94 \texteuro (Konto) 
\end{itemize}
Der ausführliche Bericht liegt dem Protokoll als Anlage bei.
\subsection{Kassenprüfung}
Es berichtet Roland Volkmann. Der Bericht der Kassenprüfer ist dem Protokoll beigefügt.
\subsection{Entlastung}
Antrag wird gestellt, den Vorstand zu entlasten. Antrag wird einstimmig angenommen.

\section{Neuwahlen}

\subsection{Vorsitzender}
Philippe Käufer kandidiert zum Vorsitzenden.
\begin{table}[H]
\begin{tabularx}{\textwidth}{XXX}
Ja & Nein & Enthaltungen / Ungültig\\
\toprule
14 & 0 & 0\\
\end{tabularx}
\caption{Geheime Wahl zum Vorsitzenden}
\end{table}
Der Gewählte nimmt das Amt an. Damit bleibt Philippe Käufer Vorsitzender.

\subsection{Stellvertretende Vorsitzende}
\subsubsection{Wahlverfahren}

Es wird beantragt, aus den Vorgeschlagenen Kandidaten vier Stellvertretende Vorsitzende zu wählen. Die Wahl soll geheim per Zustimmungswahl erfolgen.
\begin{table}[H]
\begin{tabularx}{\textwidth}{XXX}
Dafür & Dagegen & Enthaltungen\\
\toprule
12 & 0 & 2 \\
\end{tabularx}
\caption{Wahlverfahren für Stellvertretende Vorsitzende}
\end{table}
Das vorgeschlagene Wahlverfahren ist somit beschlossen.

\subsubsection{Wahlen}

Als Kandidaten werden Vorgeschlagen:
\begin{itemize}
\item Bruno Baumgart-Hageneder
\item Yannik Enss
\item Thomas Renger
\item Adrian Reyer
\item Thomas Rother
\end{itemize}

\begin{table}[H]
\begin{tabularx}{\textwidth}{XX}
Kandidat & Stimmen \\
\toprule
Bruno Baumgart-Hageneder  & 14 \\
Adrian Reyer & 14 \\
Thomas Renger & 9 \\
Yannik Enss  & 8 \\
Thomas Rother & 8 \\
\end{tabularx}
\caption{Geheime Zustimmungswahl der stellvertretenden Vorsitzenden}
\end{table}
Wegen des durch gleiche Stimmenzahl doppelt belegten vierten Platzes kommt es zur Stichwahl zwischen Yannik Enss und Thomas Rother.
\begin{table}[H]
\begin{tabularx}{\textwidth}{XX}
Kandidat & Stimmen \\
\toprule
Thomas Rother & 8 \\
Yannik Enss & 5 \\
\toprule
Enthaltung & 1 \\
\end{tabularx}
\caption{Stichwahl für den vierten stellvertretenden Vorsitzenden}
\end{table}
Bruno Baumgart-Hageneder, Thomas Renger, Adrian Reyer und Thomas Rother sind somit gewählt. Die Gewählten nehmen die Ämter an.
\subsection{Schatzmeister}
Christoph Altrock kandidiert zum Schatzmeister.
\begin{table}[H]
\begin{tabularx}{\textwidth}{XXX}
Dafür & Dagegen & Enthaltungen\\
\toprule
13 & 0 & 1 \\
\end{tabularx}
\caption{Geheime Wahl zum Schatzmeister}
\end{table}
Der Gewählte nimmt das Amt an.
\subsection{Kassenprüfer}
Zur Wahl stellen sich:
\begin{itemize}
\item Yannik Enss
\item Joachim Ernst
\item Roland Volkmann
\end{itemize}
Die Abstimmung erfolg per Handzeichen für die Gesamtliste.
\begin{table}[H]
\begin{tabularx}{\textwidth}{XXX}
Dafür & Dagegen & Enthaltungen\\
\toprule
14 & 0 & 0\\
\end{tabularx}
\caption{Wahl der Kassenprüfer}
\end{table}
Die Gewählten nehmen die Ämter an.
\clearpage

\section{Sonstiges}

Der Vorstand wird nach den Zielen für das nächste Geschäftsjahr befragt.
\begin{itemize}
\item Häufiger Treffen (12 Vorstandssitzungen sind geplant)
\item Themen aufteilen und Projekte gründen:
\begin{itemize}
\item Öffentlichkeitsarbeit
\item Technik
\item Politik
\end{itemize}
\item Mitglieder und Community in die Projekte einbeziehen.
\end{itemize}
Es folgt eine Aussprache über die Kommunikationskanäle, diese wird jedoch ohne Beschlussfassung in die Projektgruppen vertagt.
\section{Ende}
Die Sitzung wird um 12:20 geschlossen.
\vfill
\mbox{}\\
Für die Richtigkeit:\\
\\
\\
Stuttgart, 28. April 2018\\
Stuttgart, Pungenday, 45. Discord 3184 \\
\\
\\
\hfill Versammlungsleiter \hfill Wahlleiter \hfill Protokollant \hfill Vorsitzender \hfill

\end{document}