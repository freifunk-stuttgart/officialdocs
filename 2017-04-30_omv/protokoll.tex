\documentclass[a4paper]{scrartcl}
\usepackage[utf8]{inputenc}
\usepackage{caption}
\usepackage{pdfpages}
\usepackage[ngerman]{babel}
\usepackage{url}
\usepackage{ltxtable}
\usepackage{booktabs}
\usepackage{textcomp}
\usepackage{float}
\parindent0pt
\date{30. April 2017} 

\title{Ordentliche Mitgliederversammlung des Freifunk Stuttgart e.V.}

\begin{document}

\maketitle

\tableofcontents

\clearpage

\listoftables

\clearpage

\section{Begrüßung}
Thomas Renger eröffnet die Versammlung um 10:27 Uhr, Bruno Baumgart-Hageneder begrüßt den Verein in Göppingen.

\section{Feststellung ordnungsgemäße Einberufung, Beschlussfähigkeit, Tagesordnung}
\subsection{Ladung}
Es wird festgestellt, dass ordnungsgemäß und rechtzeitig geladen wurde. Alle Anwesenden bestätigen, eine fernschriftliche Einladung erhalten zu haben.

\subsection{Feststellung der Beschlussfähigkeit}
\begin{itemize}
\item es sind 16 Personen anwesend.
\item es sind 15 von 53 stimmberechtigten Vereinsmitgliedern anwesend.
\item 15\% der Vereinsmitglieder sind für die Beschlussfähigkeit nötig, die Bedingung ist folglich erfüllt.
\end{itemize}
\subsection{Gäste} 
Die Vereinsmitglieder werden befragt, ob Gäste beim Treffen zugelassen sein sollen. Darüber wird per Handzeichen abgestimmt.
\begin{table}[H]
	\begin{tabularx}{\textwidth}{XXX}
		Dafür & Dagegen & Enthaltungen\\
		\toprule
		14 & 0 & 1\\
	\end{tabularx}
	\caption{Zulassung von Gästen}
\end{table}
\subsection{Tagesordnung} 
Die vorläufige Tagesordnung wird einstimmig angenommen.

Ferner gilt die Geschäftsordnung laut github-Repository\footnote{\url{https://github.com/freifunk-stuttgart/officialdocs/blob/master/go_gv/go_gv.pdf}} fort.

\section{Wahl der Versammlungsämter}
\begin{itemize}
\item Für das Amt des Versammlungsleiters kandidiert Thomas Renger. Der Vorschlag wird einstimmig angenommen, der Gewählte nimmt das Amt an.
\item Für das Amt des Protokollanten kandidiert Hannes Zechmann. Der Vorschlag wird einstimmig angenommen, der Gewählte nimmt das Amt an.
\item Für das Amt der Wahlleiterin kandidiert Esther Hugo. Es wird per Handzeichen über die Kandidatur abgestimmt.\\
\begin{table}[H]
	\begin{tabularx}{\textwidth}{XXX}
		Dafür & Dagegen & Enthaltungen\\
		\toprule
		13 & 0 & 2\\
	\end{tabularx}
	\caption{Bestimmung der Wahlleiterin}
\end{table}
Die Gewählte nimmt das Amt an.
\item Lukas Mocek wird durch die Wahlleiterin als Wahlhelfer bestellt. Er nimmt das Amt an.
\end{itemize}

\section{Tätigkeitsberichte Vorstand und Kassenprüfer, Entlastung}
\subsection{Vorstandsbericht}
Es berichten Christoph Altrock, Wilhelm Humerez, Philippe Käufer und Thomas Renger.
\subsubsection{Dinge, die erreicht wurden}
\begin{itemize}
\item Es wurden 9 Vorstandssitzungen abgehalten, deren Protokolle einsehbar sind.
\item Auch 2016 wurde die Veranstaltung ,,Flash in den Mai'' im shackspace abgehalten.
\item Die Netzkapazität konnte erheblich erweitert werden unter anderem durch den Aufbau eines vereinseigenen Gateways (Kosten: monatlich mindestens 60, maximal 350 \texteuro).
\item Es wurden neue Netzwerkverbindungen über den Freifunk Rheinland geschaffen.
\item Ein Eventboxteam zur Koordination von Verleih und Betrieb der vereinseigenen mobilen Freifunkhardware wurde eingerichtet.
\item Die Koordination des Kontakts mit Flüchtlingshilfe und sozialen Einrichtungen wurde mithilfe eines Ticketsystems neu organisiert.
\item Es wurden Vorträge organisiert und Veranstaltungen mit Netzwerk versorgt. Unter anderem:
\begin{itemize}
\item Vorträge im evangelischen Medienhaus
\item Umsonst und draußen
\item  barcamp Stuttgart
\item  Open 2016
\item  TEDx Stuttgart
\end{itemize}
\end{itemize}
\subsubsection{Dinge, die nicht erreicht wurden}
Es wird angemerkt, dass auf der letzten Mitgliederversammlung beschlossen wurde eine Projektgruppe zu gründen, die erarbeitet, wie die Ausrichtung des Vereins in Zukunft aussehen soll. Dies wurde versäumt und soll demnächst nachgeholt werden soll
\subsection{Bericht des Schatzmeisters}
Es berichtet Christoph Altrock. Die wichtigsten Eckdaten:
\begin{itemize}
\item Gesamt-Ertrag: 4377,43 \texteuro
\item Gesamt-Aufwand: 2690,74 \texteuro
\item Netto Ertrag: 1686,69 \texteuro
\item Summe Bestandskonten: 4236,16 \texteuro
\end{itemize}
Der ausführliche Bericht liegt dem Protokoll als Anlage bei.
\subsection{Kassenprüfung}
Es berichtet Roland Volkmann. Der Bericht der Kassenprüfer ist dem Protokoll beigefügt.
\subsection{Entlastung}
Antrag wird gestellt, den Vorstand zu entlasten. Antrag wird einstimmig angenommen.

\section{Neuwahlen}

Weitere Vereinsmitglieder treffen ein. Für die folgenden Abstimmungen sind 17 Mitglieder anwesend.

\subsection{Vorsitzender}
Thomas Renger und Philippe Käufer kandidieren zum Vorsitzenden.
\begin{table}[H]
	\begin{tabularx}{\textwidth}{XXX}
		Renger & Käufer & Enthaltungen / Ungültig\\
		\toprule
		2 & 15 & 0\\
	\end{tabularx}
	\caption{Geheime Wahl zum Vorsitzenden}
\end{table}
Damit ist Philippe Käufer neuer Vorsitzender. Der Gewählte nimmt das Amt an.

\subsection{Stellvertretende Vorsitzende}
Als Kandidaten Vorgeschlagen werden Thomas Renger, Adrian Reyer, Willhelm Humerez, Bruno Baumgart-Hageneder und Joachim Ernst.\\
Es wird beantragt, dass aus den Vorgeschlagenen 3 Stellvertreter gewählt werden sollen.
\begin{table}[H]
	\begin{tabularx}{\textwidth}{XXX}
		Dafür & Dagegen & Enthaltungen\\
		\toprule
		14 & 0 & 3\\
	\end{tabularx}
	\caption{Wahlverfahren für Stellvertretende Vorsitzende}
\end{table}
Die Wahl erfolgt geheim per Zustimmungswahl.\\
\begin{table}[H]
	\begin{tabularx}{\textwidth}{XXXXX}
		Baumgart-Hageneder & Ernst & Humerez & Renger & Reyer\\
		\toprule
		12 & 9 & 7 & 10 & 16\\
	\end{tabularx}
	\caption{Geheime Zustimmungswahl der stellvertretenden Vorsitzenden}
\end{table}
Bruno Baumgart-Hageneder, Thomas Renger und Adrian Reyer sind somit gewählt. Die Gewählten nehmen die Ämter an.
\subsection{Schatzmeister}
Christoph Altrock kandidiert zum Schatzmeister.
\begin{table}[H]
	\begin{tabularx}{\textwidth}{XX}
		Dafür & Dagegen / Ungültig\\
		\toprule
		17 & 0\\
	\end{tabularx}
	\caption{Geheime Wahl zum Schatzmeister}
\end{table}
Der Gewählte nimmt das Amt an.
\subsection{Kassenprüfer}
Zur Wahl stehen Willhelm Humerez, Roland Volkmann, Hannes Zechmann, Joachim Ernst. Die Abstimmung erfolg per Handzeichen für die Gesamtliste.
\begin{table}[H]
	\begin{tabularx}{\textwidth}{XXX}
		Dafür & Dagegen & Enthaltungen\\
		\toprule
		16 & 0 & 1\\
	\end{tabularx}
	\caption{Wahl der Kassenprüfer}
\end{table}
Die Gewählten nehmen die Ämter an.
\clearpage
\section{Anträge}
Es wird beantragt, die Protokolle der Mitgliedsversammlungen zukünftig auch per E-Mail zu verteilen. Abstimung per Handzeichen.\\
\begin{table}[H]
	\begin{tabularx}{\textwidth}{XXX}
		Dafür & Dagegen & Enthaltungen\\
		\toprule
		11 & 0 & 6\\
	\end{tabularx}
	\caption{Verteilung des Protokolls per E-Mail}
\end{table}
Die E-Mail soll per Bcc, bzw. bei vorhandenem PGP-Schlüssel einzeln verteilt werden.

\section{Sonstiges}

(keine Punkte)\\

\section{Ende}

Die Sitzung wird um 12:09 geschlossen.

\vfill
\mbox{}\\
Für die Richtigkeit:\\
\\
\\
Stuttgart, \today\\
Stuttgart, Boomtime, 47. Discord 47 3183 \\
\\
\\
\hfill Versammlungsleiter \hfill Wahlleiter \hfill Protokollant \hfill Vorsitzender \hfill

\end{document}