\documentclass[a4paper]{scrartcl}
\usepackage[utf8]{inputenc}
\usepackage{caption}
\usepackage{pdfpages}
\usepackage[ngerman]{babel}
\usepackage{url}
\usepackage{ltxtable}
\usepackage{booktabs}
\usepackage{textcomp}
\usepackage{float}
\parindent0pt
\date{10. Mai 2019}

\title{Ordentliche Mitgliederversammlung des Freifunk Stuttgart e.V.}

\begin{document}

\maketitle

\tableofcontents

\clearpage

\listoftables

\clearpage

\section{Begrüßung}
Philippe Käufer eröffnet die Versammlung um 20:02 Uhr.

\section{Feststellung ordnungsgemäße Einberufung, Beschlussfähigkeit, Tagesordnung}
\subsection{Ladung}
Es wird festgestellt, dass ordnungsgemäß und rechtzeitig geladen wurde. Alle Anwesenden bestätigen, eine fernschriftliche Einladung erhalten zu haben.

\subsection{Feststellung der Beschlussfähigkeit}
\begin{itemize}
\item es sind 14 Personen anwesend.
\item es sind 14 von 54 stimmberechtigten Vereinsmitgliedern anwesend.
\item 15\% der Vereinsmitglieder sind für die Beschlussfähigkeit nötig, die Bedingung ist folglich erfüllt.
\end{itemize}
\subsection{Gäste}
Die Vereinsmitglieder werden befragt, ob Gäste beim Treffen zugelassen sein sollen. Darüber wird per Handzeichen abgestimmt. Die Zulassung von Gästen wird einstimmig beschlossen.

\subsection{Tagesordnung}
Die vorläufige Tagesordnung wird einstimmig angenommen.

Ferner gilt die Geschäftsordnung laut github-Repository\footnote{\url{https://github.com/freifunk-stuttgart/officialdocs/blob/master/go_gv/go_gv.pdf}} fort.

\section{Wahl der Versammlungsämter}
\begin{itemize}
\item Für das Amt des Versammlungsleiters kandidiert Philippe Käufer. Der Vorschlag wird einstimmig angenommen, der Gewählte nimmt das Amt an.
\item Für das Amt des Protokollanten kandidiert Thomas Renger. Der Vorschlag wird mit einer Gegenstimme angenommen, der Gewählte nimmt das Amt an.
\item Für das Amt des Wahlleiters kandidiert Frank Steiner. Der Vorschlag wird einstimmig angenommen, der Gewählte nimmt das Amt an.
\end{itemize}

\section{Tätigkeitsberichte Vorstand und Kassenprüfer, Entlastung}
\subsection{Vorstandsbericht}
Es berichtet Philippe Käufer.

\subsubsection{Community und Veranstaltungen}
\begin{itemize}
\item Umsonst und Draußen
\item Verleih der Event-Hardware (nach wie vor seltener als geplant)
\item Unterstützung der Community mit Finanzierung (z.B.) Stand beim Stadtfest Göppingen
\item Auftreten als Ansprech- und Vertragspartner
\item Freifunkseminar im Evangelischen Medienhaus
\end{itemize}

\subsubsection{Politik}

\begin{itemize}
\item Stellungnahme bei der Anhörung der CDU Lantagsfraktion zu National Roaming
\item Dabei auchi: Lobbyarbeit für die Gemeinnützigkeit der Freifunkvereine
\end{itemize}

\subsubsection{Kontakt zu den Gemeinden}

\begin{itemize}
\item Göppingen: Internet ins Landratsamt
\item Frickenhausen
\item Backnang: Angebot für den Bahnhof, dort ist aber das ganze Projekt pausiert
\item Esslingen: Kooperation mit der Gemeinde, Ausbau der östlichen Altstadt
\end{itemize}

In diesem Zusammenhang erhält Patrick Cervicek auf Beschluss des Vorstands vom Verein eine Auszeichnung für außergewöhnliches Engagement.
\\
Außerdem wurden insgesamt acht Vorstandssitzungen abgehalten.

\subsection{Bericht des Schatzmeisters}
Es berichtet Christoph Altrock. Die wichtigsten Eckdaten:
\begin{itemize}
\item Gesamt-Ertrag:  6292,00 \texteuro
\item Gesamt-Aufwand:  3153,36 \texteuro
\item Netto Ertrag:  3138,64 \texteuro
\item Summe Bestandskonten: 117,64 \texteuro  (Kasse) + 6798,29 \texteuro  (Konto) 
\end{itemize}
Der ausführliche Bericht liegt dem Protokoll als Anlage bei.
\subsection{Kassenprüfung}
Es berichtet Roland Volkmann. Der Bericht der Kassenprüfer ist dem Protokoll beigefügt.
\subsection{Entlastung}
Antrag wird gestellt, den Vorstand zu entlasten. Antrag wird einstimmig angenommen.

\section{Bericht zum aktuellen Stand der Gemeinnützigkeit von Freifunk-Initiativen}

(der vorgesehene Vortrag von Thomas Rother entfällt wegen Krankheit und wird zu einem späteren Zeitpunkt nachgeholt)

\clearpage
\section{Neuwahlen}
\subsection{Vorsitzender}
Bruno Baumgart-Hageneder und Yannik Enss kandidiert zum Vorsitzenden.
\begin{table}[H]
\begin{tabularx}{\textwidth}{XXX}
Baumgart-Hageneder & Enss & Enthaltungen / Ungültig\\
\toprule
13 & 1 & 0\\
\end{tabularx}
\caption{Geheime Wahl zum Vorsitzenden}
\end{table}
Der Gewählte nimmt das Amt an. Neuer Vorsitzender ist somit Bruno Baumgart-Hageneder.

\subsection{Schatzmeister}
Christoph Altrock kandidiert zum Schatzmeister.
\begin{table}[H]
\begin{tabularx}{\textwidth}{XXX}
Dafür & Dagegen & Enthaltungen\\
\toprule
14 & 0 & 0 \\
\end{tabularx}
\caption{Geheime Wahl zum Schatzmeister}
\end{table}
Der Gewählte nimmt das Amt an.

\subsection{Stellvertretende Vorsitzende}
\subsubsection{Wahlverfahren}

Es wird beantragt, wie bereits im vorherigen Jahr, aus den Vorgeschlagenen Kandidaten vier Stellvertretende Vorsitzende zu wählen. Die Wahl soll geheim per Zustimmungswahl erfolgen.
\begin{table}[H]
\begin{tabularx}{\textwidth}{XXX}
Dafür & Dagegen & Enthaltungen\\
\toprule
14 & 0 & 0 \\
\end{tabularx}
\caption{Wahlverfahren für Stellvertretende Vorsitzende}
\end{table}
Das vorgeschlagene Wahlverfahren ist somit beschlossen.

\subsubsection{Wahlen}

Als Kandidaten werden Vorgeschlagen:
\begin{itemize}
\item Nico Böhr
\item Yannik Enss
\item Thomas Renger
\item Adrian Reyer
\item Thomas Rother
\end{itemize}

\begin{table}[H]
\begin{tabularx}{\textwidth}{XX}
Kandidat & Stimmen \\
\toprule
Adrian Reyer & 12 \\
Thomas Rother &  12 \\
Yannik Enss & 11 \\
Nico Böhr & 10 \\
Thomas Renger & 10 \\
Enthaltung & 1 \\
\end{tabularx}
\caption{Geheime Zustimmungswahl der stellvertretenden Vorsitzenden}
\end{table}
Wegen des durch gleiche Stimmenzahl doppelt belegten vierten Platzes kommt es zur Stichwahl zwischen Nico Böhr und Thomas Renger.
\begin{table}[H]
\begin{tabularx}{\textwidth}{XX}
Kandidat & Stimmen \\
\toprule
Nico Böhr & 9 \\
Thomas Renger & 4 \\
Enthaltung & 1 \\
\end{tabularx}
\caption{Stichwahl für den vierten stellvertretenden Vorsitzenden}
\end{table}
Nico Böhr, Yannik Enss, Adrian Reyer und Thomas Rother sind somit gewählt. Die Gewählten nehmen die Ämter an. Thomas Rother wird dabei wie zuvor vereinbart durch Christoph Altrock vertreten.

\subsection{Kassenprüfer}
Zur Wahl stellen sich:
\begin{itemize}
\item Marvin Gaube
\item Philippe Käufer
\item Roland Volkmann
\end{itemize}
Die Abstimmung erfolg per Handzeichen für die Gesamtliste.
\begin{table}[H]
\begin{tabularx}{\textwidth}{XXX}
Dafür & Dagegen & Enthaltungen\\
\toprule
14 & 0 & 0\\
\end{tabularx}
\caption{Wahl der Kassenprüfer}
\end{table}
Die Gewählten nehmen die Ämter an.

\subsection{Vernichtung der Wahlzettel}
Es wird beantragt, die Wahlzettel der Mitgliederversammlung nach der Ausfertigung des Protokolls zu vernichten.
\begin{table}[H]
\begin{tabularx}{\textwidth}{XXX}
Dafür & Dagegen & Enthaltungen\\
\toprule
9 & 3 & 2\\
\end{tabularx}
\caption{Vernichtung der Wahlzettel}
\end{table}
Der Antrag ist damit angenommen.

\subsection{Vernichtung der Wahlzettel vergangener Mitgliederversammlungen}
Es wird weiterhin beantragt, auch die bisher archivierten Wahlzettel der vergangenen Mitgliederversammlung zu vernichten.
\begin{table}[H]
\begin{tabularx}{\textwidth}{XXX}
Dafür & Dagegen & Enthaltungen\\
\toprule
14 & 0 & 0\\
\end{tabularx}
\caption{Vernichtung alter Wahlzettel}
\end{table}
Der Antrag ist damit angenommen.

\clearpage
\section{Anträge}

\subsection{Satzungsänderungsanträge von Yannik Enss}

Es werden drei im wesentlichen gleichlautende Satzungsänderungsanträge eingebracht.

\subsubsection{Begründung}

Aktuell ist in der Vereinssatzung der Verbleib des Vereinsvermögens nach Auflösung nicht klar geregelt. Für gemeinnützige Vereine ist gesetzlich erforderlich dass das Vereinsvermögen keinen nicht-gemeinnützigen Organisationen oder Personen zufallen darf, auch nicht nach Auflösung des Vereins.
Daher beantrage ich folgende Änderung:

\subsubsection{§11 Abschnitt f) (aktuelle Fassung)}

,,Die Mitgliederversammlung fasst Beschlüsse im Allgemeinen mit einfacher Mehrheit der abgegebenen gültigen Stimmen; Stimmenthaltungen bleiben daher außer Betracht. Zur Änderung der Satzung ist eine Mehrheit von zwei Drittel der abgegebenen gültigen Stimmen, zur Auflösung des Vereins eine solche von drei Vierteln erforderlich. Anträge auf Änderung oder Ergänzung der Satzung müssen spätestens am 7. Tag vor Zusammentritt beim Vorstand eingereicht werden und den Stimmberechtigten spätestens am 4. Tag vor Zusammentritt der Versammlung zugänglich sein; die Abstimmung darüber ist nur dann zulässig, wenn der Antrag selbst den Wortlaut der Satzung ausdrücklich ändert oder ergänzt.''

\subsubsection{wird wie folgt geändert}

,,Die Mitgliederversammlung fasst Beschlüsse im Allgemeinen mit einfacher Mehrheit der abgegebenen gültigen Stimmen; Stimmenthaltungen bleiben daher außer Betracht. Zur Änderung der Satzung ist eine Mehrheit von zwei Drittel der abgegebenen gültigen Stimmen erforderlich. Anträge auf Änderung oder Ergänzung der Satzung müssen spätestens am 7. Tag vor Zusammentritt beim Vorstand eingereicht werden und den Stimmberechtigten spätestens am 4. Tag vor Zusammentritt der Versammlung zugänglich sein; die Abstimmung darüber ist nur dann zulässig, wenn der Antrag selbst den Wortlaut der Satzung ausdrücklich ändert oder ergänzt.''

\subsubsection{Es wird §13 mit dem folgenden Inhalt hinzugefügt}

,,§13 Auflösung, Anfall des Vereinsvermögens\\
Zur Auflösung des Vereins ist eine Mehrheit von 3/4 der abgegebenen gültigen Stimmen der Mitgliederversammlung erforderlich. Bei Auflösung des Vereins oder Wegfall steuerbegünstigter Zwecke fällt das Vermögen an den Freifunk Rheinland e.V., Hirzenrott 2-4, 52076 Aachen, der es unmittelbar und ausschließlich für gemeinnützige Zwecke zu verwenden hat. Als Liquidatoren werden die im Amt befindlichen vertretungsberechtigten Vorstandsmitglieder bestimmt, soweit die Mitgliederversammlung nichts anderes abschließend beschließt.''

\subsubsection{Diskussion}

Die Anträge werden verlesen. Da sie bis auf den Begünstigten in §13 identisch sind, findet eine Diskussion über alle Anträge gemeinsam statt. Die Begünstigten in den drei Anträgen sind im einzelnen:
\begin{itemize}
\item Freifunk Rheinland e.V., Hirzenrott 2-4, 52076 Aachen
\item Förderverein Freie Netzwerke e.V., Mühlenstr. 8a, 14167 Berlin
\item shack e.V., Ulmer Str. 255, 70327 Stuttgart
\end{itemize}

\subsubsection{Abstimmung}
\begin{table}[H]
	\begin{tabularx}{\textwidth}{XXX}
		Dafür & Dagegen & Enthaltungen\\
		\toprule
		0 & 13 & 1\\
	\end{tabularx}
	\caption{Satzungsänderungsantrag shack}
\end{table}
Der Antrag ist abgelehnt.
\\
\begin{table}[H]
	\begin{tabularx}{\textwidth}{XXX}
		Dafür & Dagegen & Enthaltungen\\
		\toprule
		0 & 12 & 2\\
	\end{tabularx}
	\caption{Satzungsänderungsantrag Freifunk Rheinland}
\end{table}
Der Antrag ist abgelehnt.
\\
\begin{table}[H]
	\begin{tabularx}{\textwidth}{XXX}
		Dafür & Dagegen & Enthaltungen\\
		\toprule
		0 & 12 & 2\\
	\end{tabularx}
	\caption{Satzungsänderungsantrag Förderverein Freie Netzwerke}
\end{table}
Der Antrag ist abgelehnt.
\\
Es wird beantragt, den Vorstand zu beauftragen, die Satzung in Hinblick auf die Tauglichkeit für eine Gemeinnützigkeit zu Überprüfen und gegebenenfalls fachkundige externe Hilfe hinzuzuziehen.
\begin{table}[H]
	\begin{tabularx}{\textwidth}{XXX}
		Dafür & Dagegen & Enthaltungen\\
		\toprule
		14 & 0 & 0\\
	\end{tabularx}
	\caption{Antrag Überprüfung der Satzung}
\end{table}
Der Antrag ist angenommen.

\subsection{Antrag von Marvin Gaube Sponsored AS}
\subsubsection{Antragstext}
Die Mitgliederversammlung möge beschließen, dass der Verein beim Vorliegen der Voraussetzungen ein sogenanntes ,,Sponsored AS'' und Provider Independent IPv6-Space beantragt/beschafft.
\subsubsection{Diskussion}
Es folgt eine Ausufernde Diskussion, in der insbesondere Wert darauf gelegt wird, dass die bisherige Dezentralität z.B. der Gateways nicht aufgegeben werden soll.
\subsubsection{Antrag zur Geschäftsordnung auf Einholung eines Meinungsbildes}
Yannik Enss stellt Antrag zur Geschäftsordnung, die Diskussion mit einem Meinungsbild abzuschließen.\\
Adrian Reyer erhebt Gegenrede, es lägen noch nicht alle Argumente vor.
\begin{table}[H]
	\begin{tabularx}{\textwidth}{XXX}
		Dafür & Dagegen & Enthaltungen\\
		\toprule
		5 & 6 & 3\\
	\end{tabularx}
	\caption{GO-Antrag Meinungsbild Sponsored AS}
\end{table}
Der Antrag ist abgelehnt.

\subsubsection{Antrag zur Geschäftsordnung auf Schließung der Rednerliste}
Yannik Enss stellt Antrag zur Geschäftsordnung, die Rednerliste zu schließen.\\
Adrian Reyer erhebt formale Gegenrede.
\begin{table}[H]
	\begin{tabularx}{\textwidth}{XXX}
		Dafür & Dagegen & Enthaltungen\\
		\toprule
		2 & 0 & 12\\
	\end{tabularx}
	\caption{GO-Antrag Rednerliste Sponsored AS}
\end{table}
Der Antrag ist angenommen.

\subsubsection{Abstimmung}
Albrecht Rebmann beantragt, den Antrag in geheimer Wahl abzustimmen, kann dafür aber nicht die erforderlichen 5\% Zustimmung gewinnen. Die Abstimmung erfolgt per Handzeichen.
\begin{table}[H]
	\begin{tabularx}{\textwidth}{XXX}
		Dafür & Dagegen & Enthaltungen\\
		\toprule
		6 & 3 & 5\\
	\end{tabularx}
	\caption{Antrag Sponsored AS}
\end{table}
Der Antrag ist somit angenommen.

\subsection{Antrag von Yannik Enss: }
\subsubsection{Antragstext}
Aufgrund des Finanzprüfungs-Berichtes sollte der Mindestbetrag von Fördermitgliedern festgelegt werden. Hiermit wird beantragt, eine Beitragsordnung zu verabschieden, welche die Mindesthöhe des Jahresbeitrags für Fördermitglieder auf 60 € festgelegt.
\subsubsection{Abstimmung}
\begin{table}[H]
	\begin{tabularx}{\textwidth}{XXX}
		Dafür & Dagegen & Enthaltungen\\
		\toprule
		8 & 2 & 4\\
	\end{tabularx}
	\caption{Antrag Mindestbeitrag Fördermitglieder}
\end{table}
Der Antrag ist somit angenommen.

\section{Sonstiges}

keine Punkte

\section{Ende}
Die Sitzung wird um 22:58 geschlossen.
\vfill
\mbox{}\\
Für die Richtigkeit:\\
\\
\\
Stuttgart, 10. Mai 2018\\
Stuttgart, Setting Orange, 54. Discord 3185 \\
\\
\\
\hfill Versammlungsleiter \hfill Wahlleiter \hfill Protokollant \hfill Vorsitzender \hfill

\end{document}