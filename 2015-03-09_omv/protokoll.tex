\documentclass[a4paper]{scrartcl}
\usepackage[utf8]{inputenc}
\usepackage{caption}
\usepackage{pdfpages}
\usepackage[ngerman]{babel}
\usepackage{url}
\usepackage{ltxtable}
\usepackage{booktabs}

\title{Ordentliche Mitgliederversammlung des Freifunk Stuttgart e.V.}

\begin{document}

\maketitle

\tableofcontents

\clearpage

\listoftables

\clearpage

\section{Begrüßung}
Thomas Renger eröffnet die Versammlung um 19:14 Uhr.

\section{Feststellung ordnungsgemäße Einberufung, Beschlussfähigkeit, TO}

Es wurde festgestellt, dass die Einladung fristgemäß zugegangen ist, die Versammlung beschlussfähig ist und die in der Einladung zugegangene Tagesordnung gültig ist.
Ferner gilt die GO laut github-Repository\footnote{\url{https://github.com/freifunk-stuttgart/officialdocs/blob/master/go_gv/go_gv.pdf}} fort.

\section{Wahl der Versammlungsämter}

Die Versammlungsleitung übernimmt, David Mändlen, die Wahlleitung Wilhelm Humerez, das Protokoll Thomas Renger.\\
Alle 3 Versammlungsämter werden per Akklamation mit großer Mehrheit bestätigt. Die Gewählten nehmen die Ämter an.

\clearpage
\section{Satzungsänderung}

Außer den vom Vorstand vorgeschlagenen Satzungsänderungen sind keine weiteren Satzungsänderungsvorschläge eingegangen.

\subsection{Satzungsänderungsantrag 1 (§2 Zweck des Vereins; Gemeinnützigkeit)}

\subsubsection{aktuelle Formulierung}

§2 Zweck des Vereins; Gemeinnützigkeit\\
a) Zweck des Vereins ist die Förderung und der Betrieb kabelloser und kabelgebundener Computernetzwerke, die der Allgemeinheit zugänglich sind (freie Netzwerke), insbesondere Freifunk-Netzwerke.\\
b) Der Verein verfolgt ausschließlich und unmittelbar gemeinnützige Zwecke im Sinne des Abschnitts “Steuerbegünstigte Zwecke” der Abgabenordnung.\\
c) Der Satzungszweck wird verwirklicht insbesondere durch folgende Maßnahmen:\\
1) Information der Mitglieder, der Öffentlichkeit und interessierter Kreise über freie Netzwerke, insbesondere durch das Internet und durch Vorträge, Veranstaltungen, Vorführungen und Publikationen;\\
2) Bereitstellung von Know-How über Technik und Anwendung freier Netzwerke;\\
3) Unterstützung des Betriebs freier Netzwerke;\\
4) Förderung und Unterstützung von Projekten und Initiativen, die in ähnlichen Bereichen tätig sind oder denen die Idee freier Netzwerke näher gebracht werden soll.\\
d) Der Verein ist selbstlos tätig; er verfolgt nicht in erster Linie eigenwirtschaftliche Zwecke.\\
e) Mittel des Vereins dürfen nur für die satzungsmäßigen Zwecke verwendet werden. Die Mitglieder erhalten keine Gewinnanteile und in ihrer Eigenschaft als Mitglieder auch keine sonstigen Zuwendungen aus Mitteln des Vereins. Es darf keine Person durch Ausgaben, die dem Zweck des Vereins fremd sind, oder durch unverhältnismäßige hohe Vergütungen begünstigt werden. Alle Inhaber von Vereinsämtern sind ehrenamtlich tätig.\\
f) Jeder Beschluss über die Änderung der Satzung ist vor dessen Anmeldung beim Registergericht dem zuständigen Finanzamt vorzulegen.\\

\subsubsection{ausformulierte geänderte Version}

{[}\dots{]}\\
a) Zweck  des Vereins ist die Förderung und der Betrieb kabelloser und  kabelgebundener Computernetzwerke, die der Allgemeinheit zugänglich  sind (freie Netzwerke), insbesondere Freifunk-Netzwerke in der Region Stuttgart.\\
{[}\dots{]}\\

\subsubsection{Begründung}

Mit Änderung des §2a) soll hervorgehoben werden, dass Freifunk Stuttgart allgemein Freie Funknetzwerke fördert, aber der Fokus sich auf die Region Stuttgart bezieht.

\subsubsection{Abstimmung}

Antrag wurde verlesen.

keine Aussprache gewünscht.

Der Änderungsantrag wurde einstimmig angenommen.

\clearpage
\subsection{Satzungsänderungsantrag 2 (§3 Erwerb der Mitgliedschaft)}

\subsubsection{aktuelle Formulierung}

§3 Erwerb der Mitgliedschaft\\
a) Mitglieder des Vereins sind ordentliche Mitglieder, Ehrenmitglieder und Fördermitglieder. Ordentliche Mitglieder und Ehrenmitglieder sind aktiv und in der Mitgliederversammlung stimmberechtigt.\\
b) Ordentliches Mitglied des Vereins kann jede Person werden, die sich mit den Zielen des Vereins verbunden fühlt und den Verein aktiv fördern will. Die Mitgliedschaft ist in Textform (§ 126b BGB) zu beantragen. Über den Antrag entscheidet der Vorstand. Der Antrag soll den Namen und die Anschrift des Antragstellers enthalten und angeben, wie der Antragsteller den Vereinszweck aktiv fördern will.\\
c) Fördermitglied des Vereins kann jede Person werden, die sich mit den Zielen des Vereins verbunden fühlt und den Verein finanziell und ideell unterstützen will. Die Mitgliedschaft ist in Textform (§ 126b BGB) zu beantragen. Über den Antrag entscheidet ein Vorstandsmitglied. Der Antrag soll den Namen und die Anschrift des Antragstellers enthalten.\\
d) Gegen den ablehnenden Bescheid des Vorstands, der mit Gründen zu versehen ist, kann der Antragsteller Beschwerde erheben. Die Beschwerde ist innerhalb eines Monats ab Zugang des ablehnenden Bescheids schriftlich beim Vorstand einzulegen. Über die Beschwerde entscheidet die nächste ordentliche Mitgliederversammlung.\\

\subsubsection{ausformulierte geänderte Version}

{[}\dots{]}\\
a) Mitglieder des Vereins sind ordentliche Mitglieder, Ehrenmitglieder und  Fördermitglieder. Ordentliche Mitglieder und Ehrenmitglieder sind  aktiv und in der Mitgliederversammlung stimmberechtigt.\\
b) Ordentliches Mitglied des Vereins kann jede Person werden, die sich mit den Zielen des Vereins verbunden fühlt und den Verein aktiv fördern  will. Die Mitgliedschaft ist in Textform (§ 126b BGB) zu erklären.\\
c) Fördermitglied des Vereins kann jede Person werden, die sich mit den Zielen des Vereins verbunden fühlt und den Verein finanziell und ideell  unterstützen will. Die Mitgliedschaft ist in Textform (§ 126b BGB) zu erklären.\\

[ d)  entfällt ]\\

\clearpage
\subsubsection{Begründung}

Durch die Satzungsänderung wird es neuen Mitgliedern erleichtert dem Verein beizutreten. Da Freifunk Stuttgart für jeden offen ist, der sich mit den Zielen des Vereins verbunden fühlt, bedarf es keiner gesonderten Prüfung bzw. Genehmigung durch den Vorstand. Auf Grund dieser Änderungen entfällt §3 d) komplett.

\subsubsection{Abstimmung}

Antrag wurde verlesen.

\begin{table}[h]
	\begin{tabularx}{\textwidth}{XXX}
		Dafür & Dagegen & Enthaltungen\\
		\toprule
		8 & 0 & 1\\
	\end{tabularx}
	\caption{Änderungsantrag 2}
\end{table}


\clearpage

\section{Verschiedenes }

keine Anträge

\vfill
\mbox{}\\
Für die Richtigkeit:\\
\\
\\
Stuttgart, \today\\
Stuttgart, Pungenday, 68. Chaos 3181\\
Stuttgart, Sternzeit 92789.25\\
\\
\\
\hfill Versammlungsleiter \hfill Wahlleiter \hfill Protokollant \hfill 1. Vorsitzender \hfill

\end{document}