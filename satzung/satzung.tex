% Kommentare für den Editor (TexWorks/TexMakerX)
% !TeX encoding   = utf8
% !TeX spellcheck = de-DE

% --- LaTeX Vorlage ------------------------------------------------------
% speziell für einfache Dokumente wie Praktikumsprotokolle
%
% Autor: Matthias Pospiech (matthias@pospiech.eu)
% ------------------------------------------------------------------------

% Dokumentenklasse (Koma Script) -----------------------------------------
\documentclass[%
   %draft,     % Entwurfsstadium
   final,      % fertiges Dokument
   paper=a4, paper=portrait, pagesize=auto, % Papier Einstellungen
   fontsize=11pt, % Schriftgröße
   ngerman, % Sprache 
 ]{scrreprt} % Classes: scrartcl, scrreprt, scrbook

% ~~~~~~~~~~~~~~~~~~~~~~~~~~~~~~~~~~~~~~~~~~~~~~~~~~~~~~~~~~~~~~~~~~~~~~~~
% encoding
% ~~~~~~~~~~~~~~~~~~~~~~~~~~~~~~~~~~~~~~~~~~~~~~~~~~~~~~~~~~~~~~~~~~~~~~~~

% Encoding der Dateien (sonst funktionieren Umlaute nicht)
\usepackage[utf8]{inputenc}

% Encoding der Verzeichnisse (für Pfade mit Umlauten und Leerzeichne)
\usepackage[%
   extendedchars, encoding, multidot, space,
   %filenameencoding=latin1, % Windows XP, Vista, 7
   filenameencoding=utf8,   % Linux, OS X
]{grffile}

% ~~~~~~~~~~~~~~~~~~~~~~~~~~~~~~~~~~~~~~~~~~~~~~~~~~~~~~~~~~~~~~~~~~~~~~~~
% Pakete und Stile
% ~~~~~~~~~~~~~~~~~~~~~~~~~~~~~~~~~~~~~~~~~~~~~~~~~~~~~~~~~~~~~~~~~~~~~~~~
% Schriften
% ~~~~~~~~~~~~~~~~~~~~~~~~~~~~~~~~~~~~~~~~~~~~~~~~~~~~~~~~~~~~~~~~~~~~~~~~
% Fonts Fonts Fonts
% ~~~~~~~~~~~~~~~~~~~~~~~~~~~~~~~~~~~~~~~~~~~~~~~~~~~~~~~~~~~~~~~~~~~~~~~~

% immer laden:
\usepackage[T1]{fontenc} % T1 Schrift Encoding
\usepackage{textcomp}	 % Zusätzliche Symbole (Text Companion font extension)

% ~~~~~~~~~~~~~~~~~~~~~~~~~~~~~~~~~~~~~~~~~~~~~~~~~~~~~~~~~~~~~~~~~~~~~~~~
% Symbole
% ~~~~~~~~~~~~~~~~~~~~~~~~~~~~~~~~~~~~~~~~~~~~~~~~~~~~~~~~~~~~~~~~~~~~~~~~

\usepackage{amssymb}
\usepackage{mathcomp}


%% ==== Zusammengesetzte Schriften  (Sans + Serif) =======================

%% - Latin Modern
\usepackage{lmodern}
%% -------------------

%% - Bera Schriften
%\usepackage{bera}
%% -------------------

%% - Times, Helvetica, Courier (Word Standard...)
%\usepackage{mathptmx}
%\usepackage[scaled=.90]{helvet}
%\usepackage{courier}
%% -------------------

%% - Palantino , Helvetica, Courier
%\usepackage{mathpazo}
%\usepackage[scaled=.95]{helvet}
%\usepackage{courier}
%% -------------------

%% - Charter, Bera Sans
%\usepackage{charter}\linespread{1.05}
%\renewcommand{\sfdefault}{fvs}
%\usepackage[charter]{mathdesign}



%%%% =========== Typewriter =============

%\usepackage{courier}                   %% --- Courier
%\renewcommand{\ttdefault}{cmtl}        %% --- CmBright Typewriter Font
%\usepackage[%                          %% --- Luxi Mono (Typewriter)
%   scaled=0.9
%]{luximono}



% Pakete Laden
% ~~~~~~~~~~~~~~~~~~~~~~~~~~~~~~~~~~~~~~~~~~~~~~~~~~~~~~~~~~~~~~~~~~~~~~~~
% These packages must be loaded before all others
% (primarily because they are required by other packages)
% ~~~~~~~~~~~~~~~~~~~~~~~~~~~~~~~~~~~~~~~~~~~~~~~~~~~~~~~~~~~~~~~~~~~~~~~~
\usepackage{calc}
\usepackage{fixltx2e}	% Fix known LaTeX2e bugs

\usepackage[ngerman]{babel} 			% Sprache
\usepackage[dvipsnames, table]{xcolor} 	% Farben

% ~~~~~~~~~~~~~~~~~~~~~~~~~~~~~~~~~~~~~~~~~~~~~~~~~~~~~~~~~~~~~~~~~~~~~~~~
% Bilder, Gleitumgebungen und Platzierung
% ~~~~~~~~~~~~~~~~~~~~~~~~~~~~~~~~~~~~~~~~~~~~~~~~~~~~~~~~~~~~~~~~~~~~~~~~

\usepackage[]{graphicx}					% Graphiken
\usepackage{epstopdf}		% konvertiert eps in pdf

% provides new floats and enables H float modifier option
\usepackage{float}
% Floats immer erst nach der Referenz setzen
\usepackage{flafter}
% Alel Floats werden vor der nächsten section ausgegeben
\usepackage[section]{placeins} 
%

\usepackage{tikz}
%tikz verwenden um Grafiken zu erstellen
\usepackage{pgfplots}
%pgfplots verwenden für Diagramme

% ~~~~~~~~~~~~~~~~~~~~~~~~~~~~~~~~~~~~~~~~~~~~~~~~~~~~~~~~~~~~~~~~~~~~~~~~
% Beschriftungen (captions)
% ~~~~~~~~~~~~~~~~~~~~~~~~~~~~~~~~~~~~~~~~~~~~~~~~~~~~~~~~~~~~~~~~~~~~~~~~

\usepackage{caption}
\usepackage[list=true, listformat=simple]{subcaption}

% ~~~~~~~~~~~~~~~~~~~~~~~~~~~~~~~~~~~~~~~~~~~~~~~~~~~~~~~~~~~~~~~~~~~~~~~~
% Math
% ~~~~~~~~~~~~~~~~~~~~~~~~~~~~~~~~~~~~~~~~~~~~~~~~~~~~~~~~~~~~~~~~~~~~~~~~

% Base Math Package
\usepackage[fleqn]{amsmath} 
% Warnt bei Benutzung von Befehlen die mit amsmath inkompatibel sind.
\usepackage[all, error]{onlyamsmath}
%Brüche schräg statt horizontal brechen
\usepackage{nicefrac}

% ~~~~~~~~~~~~~~~~~~~~~~~~~~~~~~~~~~~~~~~~~~~~~~~~~~~~~~~~~~~~~~~~~~~~~~~~
% Science
% ~~~~~~~~~~~~~~~~~~~~~~~~~~~~~~~~~~~~~~~~~~~~~~~~~~~~~~~~~~~~~~~~~~~~~~~~

% Einheiten und Zahlenformatierung
\usepackage{siunitx}

% ~~~~~~~~~~~~~~~~~~~~~~~~~~~~~~~~~~~~~~~~~~~~~~~~~~~~~~~~~~~~~~~~~~~~~~~~
% Tables (Tabular)
% ~~~~~~~~~~~~~~~~~~~~~~~~~~~~~~~~~~~~~~~~~~~~~~~~~~~~~~~~~~~~~~~~~~~~~~~~

\usepackage{booktabs}
\usepackage{ltxtable} % Longtable + tabularx

% ~~~~~~~~~~~~~~~~~~~~~~~~~~~~~~~~~~~~~~~~~~~~~~~~~~~~~~~~~~~~~~~~~~~~~~~~
% text related packages
% ~~~~~~~~~~~~~~~~~~~~~~~~~~~~~~~~~~~~~~~~~~~~~~~~~~~~~~~~~~~~~~~~~~~~~~~~

\usepackage{url}            % Befehl \url{...}
\usepackage{enumerate}		% Kompakte Listen

% Neue Befehle: \Centering, \RaggedLeft, and \RaggedRight, ... 
\usepackage{ragged2e}


% ~~~~~~~~~~~~~~~~~~~~~~~~~~~~~~~~~~~~~~~~~~~~~~~~~~~~~~~~~~~~~~~~~~~~~~~~
% Citations
% ~~~~~~~~~~~~~~~~~~~~~~~~~~~~~~~~~~~~~~~~~~~~~~~~~~~~~~~~~~~~~~~~~~~~~~~~

%\usepackage[
%	style=alphabetic, % Loads the bibliography and the citation style 
%	natbib=true, % define natbib compatible cite commands
%]{biblatex}	
% Other options:
%	style=numeric, % 
%	style=numeric-comp,    % [1–3, 7, 8]
%	style=numeric-verb,    % [2]; [5]; [6]

\usepackage{csquotes}

% ~~~~~~~~~~~~~~~~~~~~~~~~~~~~~~~~~~~~~~~~~~~~~~~~~~~~~~~~~~~~~~~~~~~~~~~~
% layout packages
% ~~~~~~~~~~~~~~~~~~~~~~~~~~~~~~~~~~~~~~~~~~~~~~~~~~~~~~~~~~~~~~~~~~~~~~~~
%
% Befehle für 1,5 und 2 zeilig: 
% \singlespacing, \onehalfspacing und \doublespacing
\usepackage{setspace}

% ~~~~~~~~~~~~~~~~~~~~~~~~~~~~~~~~~~~~~~~~~~~~~~~~~~~~~~~~~~~~~~~~~~~~~~~~
% Kopf und Fusszeile
% ~~~~~~~~~~~~~~~~~~~~~~~~~~~~~~~~~~~~~~~~~~~~~~~~~~~~~~~~~~~~~~~~~~~~~~~~

% Kopf und Fusszeile mit scrpage2 einstellen
\usepackage[automark, komastyle, nouppercase]{scrpage2}

% ~~~~~~~~~~~~~~~~~~~~~~~~~~~~~~~~~~~~~~~~~~~~~~~~~~~~~~~~~~~~~~~~~~~~~~~~
% pdf packages
% ~~~~~~~~~~~~~~~~~~~~~~~~~~~~~~~~~~~~~~~~~~~~~~~~~~~~~~~~~~~~~~~~~~~~~~~~

% Include pages from external PDF documents in LaTeX documents
\usepackage{pdfpages} 

% Optischer Randausgleich mit pdfTeX
\usepackage{microtype}

% Einstellungen und Layoutstile 
% ~~~~~~~~~~~~~~~~~~~~~~~~~~~~~~~~~~~~~~~~~~~~~~~~~~~~~~~~~~~~~~~~~~~~~~~~
% Colors
% ~~~~~~~~~~~~~~~~~~~~~~~~~~~~~~~~~~~~~~~~~~~~~~~~~~~~~~~~~~~~~~~~~~~~~~~~
\definecolor{sectioncolor}{RGB}{0, 0, 0}     % black

% ~~~~~~~~~~~~~~~~~~~~~~~~~~~~~~~~~~~~~~~~~~~~~~~~~~~~~~~~~~~~~~~~~~~~~~~~
% text related 
% ~~~~~~~~~~~~~~~~~~~~~~~~~~~~~~~~~~~~~~~~~~~~~~~~~~~~~~~~~~~~~~~~~~~~~~~~

%% style of URL
\urlstyle{tt}


% Keine hochgestellten Ziffern in der Fussnote (KOMA-Script-spezifisch):
\deffootnote{1.5em}{1em}{\makebox[1.5em][l]{\thefootnotemark}}

% Limit space of footnotes to 10 lines
\setlength{\dimen\footins}{10\baselineskip}

% prevent continuation of footnotes 
% at facing page
\interfootnotelinepenalty=10000 

% ~~~~~~~~~~~~~~~~~~~~~~~~~~~~~~~~~~~~~~~~~~~~~~~~~~~~~~~~~~~~~~~~~~~~~~~~
% Science
% ~~~~~~~~~~~~~~~~~~~~~~~~~~~~~~~~~~~~~~~~~~~~~~~~~~~~~~~~~~~~~~~~~~~~~~~~

\sisetup{%
	mode = math, detect-family, detect-weight,	
	exponent-product = \cdot,
	number-unit-separator=\text{\,},
	output-decimal-marker={,},
}

% ~~~~~~~~~~~~~~~~~~~~~~~~~~~~~~~~~~~~~~~~~~~~~~~~~~~~~~~~~~~~~~~~~~~~~~~~
% Citations / Style of Bibliography
% ~~~~~~~~~~~~~~~~~~~~~~~~~~~~~~~~~~~~~~~~~~~~~~~~~~~~~~~~~~~~~~~~~~~~~~~~

% Kommentar entfernene wenn biblatex geladen wird
% \input{preambel/style-biblatex}

% ~~~~~~~~~~~~~~~~~~~~~~~~~~~~~~~~~~~~~~~~~~~~~~~~~~~~~~~~~~~~~~~~~~~~~~~~
% figures, placement, floats and captions
% ~~~~~~~~~~~~~~~~~~~~~~~~~~~~~~~~~~~~~~~~~~~~~~~~~~~~~~~~~~~~~~~~~~~~~~~~

% Make float placement easier
\renewcommand{\floatpagefraction}{.75} % vorher: .5
\renewcommand{\textfraction}{.1}       % vorher: .2
\renewcommand{\topfraction}{.8}        % vorher: .7
\renewcommand{\bottomfraction}{.5}     % vorher: .3
\setcounter{topnumber}{3}        % vorher: 2
\setcounter{bottomnumber}{2}     % vorher: 1
\setcounter{totalnumber}{5}      % vorher: 3

%% ~~~ Captions ~~~~~~~~~~~~~~~~~~~~~~~~~~~~~~~~~~~~~~~~~~~~~~~~~~~~~~~~~~
% Style of captions
\DeclareCaptionStyle{captionStyleTemplateDefault}
[ % single line captions
   justification = centering
]
{ % multiline captions
% -- Formatting
   format      = plain,  % plain, hang
   indention   = 0em,    % indention of text 
   labelformat = default,% default, empty, simple, brace, parens
   labelsep    = colon,  % none, colon, period, space, quad, newline, endash
   textformat  = simple, % simple, period
% -- Justification
   justification = justified, %RaggedRight, justified, centering
   singlelinecheck = true, % false (true=ignore justification setting in single line)
% -- Fonts
   labelfont   = {small,bf},
   textfont    = {small,rm},
% valid values:
% scriptsize, footnotesize, small, normalsize, large, Large
% normalfont, ip, it, sl, sc, md, bf, rm, sf, tt
% singlespacing, onehalfspacing, doublespacing
% normalcolor, color=<...>
%
% -- Margins and further paragraph options
   margin = 10pt, %.1\textwidth,
   % width=.8\linewidth,
% -- Skips
   skip     = 10pt, % vertical space between the caption and the figure
   position = auto, % top, auto, bottom
% -- Lists
   % list=no, % suppress any entry to list of figure 
   listformat = subsimple, % empty, simple, parens, subsimple, subparens
% -- Names & Numbering
   % figurename = Abb. %
   % tablename  = Tab. %
   % listfigurename=
   % listtablename=
   % figurewithin=chapter
   % tablewithin=chapter
%-- hyperref related options
	hypcap=true, % (true, false) 
	% true=all hyperlink anchors are placed at the 
	% beginning of the (floating) environment
	%
	hypcapspace=0.5\baselineskip
}

% apply caption style
\captionsetup{
	style = captionStyleTemplateDefault % base
}

% Predefinded skip setup for different floats
\captionsetup[table]{position=top}
\captionsetup[figure]{position=bottom}


% options for subcaptions
\captionsetup[sub]{ %
	style = captionStyleTemplateDefault, % base
	skip=6pt,
	margin=5pt,
	labelformat = parens,% default, empty, simple, brace
	labelsep    = space,
	list=false,
	hypcap=false
}

% ~~~~~~~~~~~~~~~~~~~~~~~~~~~~~~~~~~~~~~~~~~~~~~~~~~~~~~~~~~~~~~~~~~~~~~~~
% layout 
% ~~~~~~~~~~~~~~~~~~~~~~~~~~~~~~~~~~~~~~~~~~~~~~~~~~~~~~~~~~~~~~~~~~~~~~~~


%% Paragraph Separation =================================
\KOMAoptions{%
   parskip=absolute, % do not change indentation according to fontsize
   parskip=false     % indentation of 1em
   % parskip=half    % parksip of 1/2 line 
}%

%% line spacing =========================================
%\onehalfspacing	% 1,5-facher Abstand
%\doublespacing		% 2-facher Abstand

%% page layout ==========================================

\raggedbottom     % Variable Seitenhoehen zulassen

% Koma Script text area layout
\KOMAoptions{%
   DIV=11,% (Size of Text Body, higher values = greater textbody)
   BCOR=5mm% (Bindekorrektur)
}%

%%% === Page Layout  Options ===
\KOMAoptions{% (most options are for package typearea)
   % twoside=true, % two side layout (alternating margins, standard in books)
   twoside=false, % single side layout 
   %
   headlines=2.1,%
}%

%\KOMAoptions{%
%      headings=noappendixprefix % chapter in appendix as in body text
%      ,headings=nochapterprefix  % no prefix at chapters
%      % ,headings=appendixprefix   % inverse of 'noappendixprefix'
%      % ,headings=chapterprefix    % inverse of 'nochapterprefix'
%      % ,headings=openany   % Chapters start at any side
%      % ,headings=openleft  % Chapters start at left side
%      ,headings=openright % Chapters start at right side      
%}%


% reloading of typearea, necessary if setting of spacing changed
\typearea[current]{last}

%% Zitate
\renewcommand*{\dictumwidth}{0.5\textwidth}

% ~~~~~~~~~~~~~~~~~~~~~~~~~~~~~~~~~~~~~~~~~~~~~~~~~~~~~~~~~~~~~~~~~~~~~~~~
% Titlepage
% ~~~~~~~~~~~~~~~~~~~~~~~~~~~~~~~~~~~~~~~~~~~~~~~~~~~~~~~~~~~~~~~~~~~~~~~~
\KOMAoptions{%
   % titlepage=true %
   titlepage=false %
}%

% ~~~~~~~~~~~~~~~~~~~~~~~~~~~~~~~~~~~~~~~~~~~~~~~~~~~~~~~~~~~~~~~~~~~~~~~~
% head and foot lines
% ~~~~~~~~~~~~~~~~~~~~~~~~~~~~~~~~~~~~~~~~~~~~~~~~~~~~~~~~~~~~~~~~~~~~~~~~

% \pagestyle{scrheadings} % Seite mit Headern
\pagestyle{scrplain} % Seiten ohne Header

% loescht voreingestellte Stile
\clearscrheadings
\clearscrplain
%
% Was steht wo...
% Bei headings:
%   % Oben aussen: Kapitel und Section
%   % Unten aussen: Seitenzahl
%   \ohead{\pagemark}
%   \ihead{\headmark}
%   \ofoot[\pagemark]{} % Außen unten: Seitenzahlen bei plain
% Bei Plain:
\cfoot[\pagemark]{\pagemark} % Mitte unten: Seitenzahlen bei plain


% Angezeigte Abschnitte im Header
% \automark[section]{chapter} %[rechts]{links}
\automark[subsection]{section} %[rechts]{links}

% ~~~~~~~~~~~~~~~~~~~~~~~~~~~~~~~~~~~~~~~~~~~~~~~~~~~~~~~~~~~~~~~~~~~~~~~~
% headings / page opening
% ~~~~~~~~~~~~~~~~~~~~~~~~~~~~~~~~~~~~~~~~~~~~~~~~~~~~~~~~~~~~~~~~~~~~~~~~
\setcounter{secnumdepth}{3}

\KOMAoptions{%
%%%% headings
   % headings=small  % Small Font Size, thin spacing above and below
   % headings=normal % Medium Font Size, medium spacing above and below
   headings=big % Big Font Size, large spacing above and below
}%

% Titelzeile linksbuendig, haengend
\renewcommand*{\raggedsection}{\raggedright} 

% ~~~~~~~~~~~~~~~~~~~~~~~~~~~~~~~~~~~~~~~~~~~~~~~~~~~~~~~~~~~~~~~~~~~~~~~~
% fonts of headings
% ~~~~~~~~~~~~~~~~~~~~~~~~~~~~~~~~~~~~~~~~~~~~~~~~~~~~~~~~~~~~~~~~~~~~~~~~
\setkomafont{sectioning}{\normalfont\sffamily} % \rmfamily
\setkomafont{descriptionlabel}{\itshape}
\setkomafont{pageheadfoot}{\normalfont\normalcolor\small\sffamily}
\setkomafont{pagenumber}{\normalfont\sffamily}

%%% --- Titlepage ---
%\setkomafont{subject}{}
%\setkomafont{subtitle}{}
%\setkomafont{title}{}

% ~~~~~~~~~~~~~~~~~~~~~~~~~~~~~~~~~~~~~~~~~~~~~~~~~~~~~~~~~~~~~~~~~~~~~~~~
% settings and layout of TOC, LOF, 
% ~~~~~~~~~~~~~~~~~~~~~~~~~~~~~~~~~~~~~~~~~~~~~~~~~~~~~~~~~~~~~~~~~~~~~~~~
\setcounter{tocdepth}{3} % Depth of TOC Display

% ~~~~~~~~~~~~~~~~~~~~~~~~~~~~~~~~~~~~~~~~~~~~~~~~~~~~~~~~~~~~~~~~~~~~~~~~
% Tabellen
% ~~~~~~~~~~~~~~~~~~~~~~~~~~~~~~~~~~~~~~~~~~~~~~~~~~~~~~~~~~~~~~~~~~~~~~~~

%%% -| Neue Spaltendefinitionen 'columntypes' |--
%
% Belegte Spaltentypen:
% l - links
% c - zentriert
% r - rechts
% p,m,b  - oben, mittig, unten
% X - tabularx Auto-Spalte

% um Tabellenspalten mit Flattersatz zu setzen, muss \\ vor
% (z.B.) \raggedright geschuetzt werden:
\newcommand{\PreserveBackslash}[1]{\let\temp=\\#1\let\\=\temp}

% Spalten mit Flattersatz und definierte Breite:
% m{} -> mittig
% p{} -> oben
% b{} -> unten
%
% Linksbuendig:
\newcolumntype{v}[1]{>{\PreserveBackslash\RaggedRight\hspace{0pt}}p{#1}}
\newcolumntype{M}[1]{>{\PreserveBackslash\RaggedRight\hspace{0pt}}m{#1}}
% % Rechtsbuendig :
% \newcolumntype{R}[1]{>{\PreserveBackslash\RaggedLeft\hspace{0pt}}m{#1}}
% \newcolumntype{S}[1]{>{\PreserveBackslash\RaggedLeft\hspace{0pt}}p{#1}}
% % Zentriert :
% \newcolumntype{Z}[1]{>{\PreserveBackslash\Centering\hspace{0pt}}m{#1}}
% \newcolumntype{A}[1]{>{\PreserveBackslash\Centering\hspace{0pt}}p{#1}}

\newcolumntype{Y}{>{\PreserveBackslash\RaggedLeft\hspace{0pt}}X}

%-- Einstellungen für Tabellen ----------
\providecommand\tablestyle{%
  \renewcommand{\arraystretch}{1.4} % Groessere Abstaende zwischen Zeilen
  \normalfont\normalsize            %
  \sffamily\small           % Serifenlose und kleine Schrift
  \centering%                       % Tabelle zentrieren
}

%--Einstellungen für Tabellen ----------

\colorlet{tablesubheadcolor}{gray!40}
\colorlet{tableheadcolor}{gray!25}
\colorlet{tableblackheadcolor}{black!60}
\colorlet{tablerowcolor}{gray!15.0}

%--Paragraph vor Kapitel--
\renewcommand{\thechapter}{\S\arabic{chapter}}

% ~~~~~~~~~~~~~~~~~~~~~~~~~~~~~~~~~~~~~~~~~~~~~~~~~~~~~~~~~~~~~~~~~~~~~~~~
% pdf packages
% ~~~~~~~~~~~~~~~~~~~~~~~~~~~~~~~~~~~~~~~~~~~~~~~~~~~~~~~~~~~~~~~~~~~~~~~~

% ~~~~~~~~~~~~~~~~~~~~~~~~~~~~~~~~~~~~~~~~~~~~~~~~~~~~~~~~~~~~~~~~~~~~~~~~
% fix remaining problems
% ~~~~~~~~~~~~~~~~~~~~~~~~~~~~~~~~~~~~~~~~~~~~~~~~~~~~~~~~~~~~~~~~~~~~~~~~



% ~~~~~~~~~~~~~~~~~~~~~~~~~~~~~~~~~~~~~~~~~~~~~~~~~~~~~~~~~~~~~~~~~~~~~~~~
% Eigene Befehle
% ~~~~~~~~~~~~~~~~~~~~~~~~~~~~~~~~~~~~~~~~~~~~~~~~~~~~~~~~~~~~~~~~~~~~~~~~
% -- new commands --
\providecommand{\abs}[1]{\lvert#1\rvert}
\providecommand{\Abs}[1]{\left\lvert#1\right\rvert}
\providecommand{\norm}[1]{\left\Vert#1\right\Vert}
\providecommand{\Trace}[1]{\ensuremath{\Tr\{\,#1\,\}}} % Trace /Spur
%

\renewcommand{\d}{\partial\mspace{2mu}} % partial diff
\newcommand{\td}{\,\mathrm{d}}	% total diff

\newcommand{\Ham}{\mathcal{H}}    % Hamilton
\newcommand{\Prob}{\mathscr{P}}    % Hamilton
\newcommand{\unity}{\mathds{1}}   % Real

\renewcommand{\i}{\mathrm{i}}   % imagin�re Einheit



% -- New Operators --
\DeclareMathOperator{\rot}{rot}
\DeclareMathOperator{\grad}{grad}
\DeclareMathOperator{\Tr}{Tr}
\DeclareMathOperator{\const}{const}
\DeclareMathOperator{\e}{e} 			% exponatial Function


\renewcommand{\labelenumi}{\alph{enumi})}
\renewcommand{\labelenumii}{\roman{enumii})}
\setlength{\parindent}{0pt}
% ~~~~~~~~~~~~~~~~~~~~~~~~~~~~~~~~~~~~~~~~~~~~~~~~~~~~~~~~~~~~~~~~~~~~~~~~
% Eigene Befehle
% ~~~~~~~~~~~~~~~~~~~~~~~~~~~~~~~~~~~~~~~~~~~~~~~~~~~~~~~~~~~~~~~~~~~~~~~~
% Silbentrennung hinzufügen als 
% Sil-ben-tren-nung 
\hyphenation{}

\listfiles % schreibt alle verwendeten Dateien in die log Datei

%% Dokument Beginn %%%%%%%%%%%%%%%%%%%%%%%%%%%%%%%%%%%%%%%%%%%%%%%%%%%%%%%%
\begin{document}
% Automatische Titelseite

\subject{Satzung}
\title{Freifunk Stuttgart e.V.\\}
\author{in der Fassung vom}
\date{2015-03-09}
\maketitle

% Manuelle Titelseite

%\begin{titlepage}
%   \mbox{}\vspace{5\baselineskip}\\
%   \sffamily\huge
%   \centering
%   % Titel
%   \LaTeXe{}-Vorlage von Matthias Pospiech
%   \vspace{3\baselineskip}\\
%   \rmfamily\Large
%   Leibniz Universität Hannover
%   \vspace{2\baselineskip}\\
%   \rmfamily\Large
%   Matthias Pospiech
%   \vspace{1\baselineskip}\\
%   \today
%\end{titlepage}


\clearpage

\null

\vfill

\dictum[Captain Jean-Luc Picard]{Make it so.}

\clearpage

\tableofcontents

% Testdokumente (auskommentieren!)
%\input{content/demo/demo.tex}
%\input{content/demo/latexexample.tex}

% in diese Datei gehört der Inhalt des Dokumentes:
\chapter{Name, Sitz, Geschäftsjahr}
\begin{enumerate}[a)]
	\item Der Verein führt den Namen „Freifunk Stuttgart“, in Langform „Freifunk Stuttgart e.V.“, und soll im Vereinsregister der Stadt Stuttgart eingetragen werden.
	\item Der Verein hat seinen Sitz in Stuttgart.
	\item Das Geschäftsjahr des Vereins ist das Kalenderjahr.
\end{enumerate}

\chapter{Zweck des Vereins; Gemeinnützigkeit}
\begin{enumerate}
	\item Zweck des Vereins ist die Förderung und der Betrieb kabelloser und kabelgebundener Computernetzwerke, die der Allgemeinheit zugänglich sind (freie Netzwerke), insbesondere Freifunk-Netzwerke in der Region Stuttgart.
	\item Der Verein verfolgt ausschließlich und unmittelbar gemeinnützige Zwecke im Sinne des Abschnitts “Steuerbegünstigte Zwecke” der Abgabenordnung.
	\item Der Satzungszweck wird verwirklicht insbesondere durch folgende Maßnahmen:
		\begin{enumerate}[1)]
			\item Information der Mitglieder, der Öffentlichkeit und interessierter Kreise über freie Netzwerke, insbesondere durch das Internet und durch Vorträge, Veranstaltungen, Vorführungen und Publikationen;
			\item Bereitstellung von Know-How über Technik und Anwendung freier Netzwerke;
			\item Unterstützung des Betriebs freier Netzwerke;
			\item Förderung und Unterstützung von Projekten und Initiativen, die in ähnlichen Bereichen tätig sind oder denen die Idee freier Netzwerke näher gebracht werden soll.
		\end{enumerate}
	\item Der Verein ist selbstlos tätig; er verfolgt nicht in erster Linie eigenwirtschaftliche Zwecke.
	\item Mittel des Vereins dürfen nur für die satzungsmäßigen Zwecke verwendet werden. Die Mitglieder erhalten keine Gewinnanteile und in ihrer Eigenschaft als Mitglieder auch keine sonstigen Zuwendungen aus Mitteln des Vereins. Es darf keine Person durch Ausgaben, die dem Zweck des Vereins fremd sind, oder durch unverhältnismäßige hohe Vergütungen begünstigt werden. Alle Inhaber von Vereinsämtern sind ehrenamtlich tätig.
	\item Jeder Beschluss über die Änderung der Satzung ist vor dessen Anmeldung beim Registergericht dem zuständigen Finanzamt vorzulegen.
\end{enumerate}

\chapter{Erwerb der Mitgliedschaft}
\begin{enumerate}
	\item Mitglieder des Vereins sind ordentliche Mitglieder, Ehrenmitglieder und Fördermitglieder. Ordentliche Mitglieder und Ehrenmitglieder sind aktiv und in der Mitgliederversammlung stimmberechtigt.
	\item Ordentliches Mitglied des Vereins kann jede Person werden, die sich mit den Zielen des Vereins verbunden fühlt und den Verein aktiv fördern  will. Die Mitgliedschaft ist in Textform (§ 126b BGB) zu erklären.
	\item Fördermitglied des Vereins kann jede Person werden, die sich mit den Zielen des Vereins verbunden fühlt und den Verein finanziell und ideell  unterstützen will. Die Mitgliedschaft ist in Textform (§ 126b BGB) zu erklären.
\end{enumerate}

\chapter{Beendigung der Mitgliedschaft}

\begin{enumerate}
	\item Die Mitgliedschaft endet:
		\begin{enumerate}[1)]
			\item mit dem Tod des Mitglieds;
			\item durch freiwilligen Austritt;
			\item durch Ausschluss aus dem Verein;
			\item bei Ausbleiben des Mitgliedsbetrags länger als 3 Monate.
		\end{enumerate}

	\item Der freiwillige Austritt erfolgt durch Erklärung gegenüber einem Mitglied des Vorstands in Textform. Der Austritt ist zum Monatsende unter Einhaltung einer Kündigungsfrist von zwei Wochen zulässig.
	\item Ein Mitglied kann, wenn es gegen die Vereinsinteressen grob verstoßen hat, durch Beschluss des Vorstands aus dem Verein ausgeschlossen werden. Vor der Beschlussfassung ist dem Mitglied unter Setzung einer angemessenen Frist Gelegenheit zu geben, sich persönlich vor dem Vorstand oder in Textform zu rechtfertigen. Eine in Textform vorliegende Stellungnahme des Betroffenen ist in der Vorstandssitzung zu verlesen. Der Beschluss über den Ausschluss ist mit
		Gründen zu versehen und dem Mitglied mittels eingeschriebenen Briefes bekannt zu machen. Gegen den Ausschließungsbeschluss des Vorstands steht dem Mitglied das Recht der Berufung an die Mitgliederversammlung zu. Die Berufung hat aufschiebende Wirkung. Die Berufung muss innerhalb einer Frist von einem Monat ab Zugang des Ausschließungsbeschlusses beim Vorstand schriftlich eingelegt werden. Ist die Berufung rechtzeitig eingelegt, so hat
		der Vorstand innerhalb von zwei Monaten die Mitgliederversammlung zur Entscheidung über die Berufung einzuberufen. Geschieht das nicht, gilt der Ausschließungsbeschluss als nicht erlassen. Macht das Mitglied von dem Recht der Berufung gegen den Ausschließungsbeschluss keinen Gebrauch oder versäumt es die Berufungsfrist, so unterwirft es sich damit dem Ausschließungsbeschluss mit der Folge, dass die Mitgliedschaft als beendet gilt.
\end{enumerate}

\chapter{Mitgliedsbeiträge}
\begin{enumerate}
	\item Von den Mitgliedern werden Beiträge erhoben. Die Höhe des Jahresbeitrags und dessen Fälligkeit werden von der Mitgliederversammlung bestimmt.
	\item Ehrenmitglieder sind von der Beitragspflicht befreit.
\end{enumerate}

\chapter{Organe}

Die Organe des Vereins sind:
\begin{enumerate}
	\item Die Mitgliederversammlung
	\item Der Vorstand
\end{enumerate}

\chapter{Der Vorstand}
\begin{enumerate}
	\item Der Vorstand des Vereins besteht aus einem Vorsitzenden, mindestens einem stellvertretenden Vorsitzenden und einem Schatzmeister. Der Vorstand kann für seine Tätigkeit eine Geschäftsordnung beschließen.
	\item Der Vorstand führt die Geschäfte des Vereins und vertritt diesen nach innen und außen.
	\item Jedes Vorstandsmitglied ist allein vertretungsberechtigt im Sinne des §26, BGB bei Rechtsgeschäften bis zu einem Höchstbetrag von 1500 Euro. Bei Rechtsgeschäften über 1500 Euro ist die Vertretung durch zwei Vorstandsmitglieder erforderlich.
	\item Die ordnungsgemäße finanzielle Geschäftsführung des Vereins wird durch die Finanzprüfer kontrolliert.
\end{enumerate}

\chapter{Wahl und Amtsdauer des Vorstands}
Der Vorstand wird von der Mitgliederversammlung auf die Dauer von 12 Monaten, vom Tage der Wahl an gerechnet, gewählt; er bleibt jedoch bis zur Neuwahl des Vorstands im Amt. Wählbar sind nur aktive Vereinsmitglieder.

\chapter{Einberufung der Mitgliederversammlung}
\begin{enumerate}
	\item  Die Versammlung wird mindestens einmal im Kalenderjahr einberufen durch die Ladung aller stimmberechtigten Mitglieder. Die Einladung erfolgt unter Einhaltung einer Frist von mindestens vier Wochen per E-Mail und muss mindestens enthalten:
		\begin{enumerate}[1)]
			\item den Anlass der Einberufung
			\item das kalendarische Datum
			\item den genauen Ort (postalische Adresse)
			\item die genaue Uhrzeit der Akkreditierung, Beginn und geplantes Ende der Versammlung
			\item die vorläufige Tagesordnung
			\item Angaben dazu, wo bereits vorliegende Anträge in Textform aufzufinden und einzusehen sind
			\item Namen und Amtsbezeichnung des Ladenden.
		\end{enumerate}
	\item Ort und Datum der Mitgliederversammlung sollen zudem in den Medien des Vereins bekannt gegeben werden.
\end{enumerate}

\chapter{Mitgliederversammlung}
\begin{enumerate}
	\item Oberstes Beschlussorgan ist die Mitgliederversammlung. In der Mitgliederversammlung hat jedes aktive Mitglied eine Stimme. Die Mitgliederversammlung tagt, sofern nicht anders beschlossen, öffentlich.
	\item Fördermitglieder haben in der Mitgliederversammlung ein Anwesenheits- und Antragsrecht, sind aber nicht stimmberechtigt.
	\item Die Mitgliederversammlung ist für folgende Angelegenheiten zuständig:
		\begin{enumerate}[1)]
			\item Genehmigung des vom Vorstand aufgestellten Haushaltsplans für das nächste Geschäftsjahr; Entgegennahme des Jahresberichts des Vorstands; Entlastung des Vorstands;
			\item Festsetzung der Höhe und der Fälligkeit des Jahresbeitrags;
			\item Wahl und Abberufung der Mitglieder des Vorstands;
			\item Bestellung von Finanzprüfern;
			\item Beschlussfassung über Änderung der Satzung und über die Auflösung des Vereins;
			\item Beschlussfassung über die Beschwerde gegen einen Ausschließungsbeschluss des Vorstands:
			\item Ernennung von Ehrenmitgliedern.
		\end{enumerate}
\end{enumerate}

\chapter{Beschlussfassung der Mitgliederversammlung}
\begin{enumerate}
	\item Bis die Versammlungsleitung gewählt ist, leitet der Vorsitzende die Mitgliederversammlung; ist dieser verhindert oder lehnt die Versammlungsleitung ab, richtet sich seine Vertretung nach der Vertretungsregelung im Vorstand. Steht aus rechtlichen oder tatsächlichen Gründen kein Stellvertreter zur Verfügung, dann leitet bis zur Wahl des ersten Versammlungsleiters das Mitglied der Mitgliederversammlung die Tagung, das am längsten Mitglied des Vereins ist. Im
		Zweifel entscheidet die Reihenfolge der Mitgliedsnummer.
	\item Der vorläufige Versammlungsleiter kann die Tagung der Mitgliederversammlung erst nach dem Zeitpunkt eröffnen, für den die Versammlung geladen war.
	\item Die Mitgliederversammlung wählt ihre Versammlungsleitung, die mindestens aus einem Versammlungsleiter, einem Wahlleiter und einem Protokollanten besteht; bei diesen Wahlen wird offen abgestimmt, sofern sich auf ausdrückliches Befragen kein Widerspruch erhebt. Nach der Wahl des ersten Versammlungsleiters, hat der vorläufige Versammlungsleiter ihm die Leitung der Versammlung zu übergeben.
	\item Die Mitgliederversammlung gibt sich eine Geschäftsordnung. Diese gilt bis sie ausdrücklich geändert oder ergänzt wird. Dies wird offen abgestimmt.
	\item Die Mitgliederversammlung ist beschlussfähig, wenn mindestens 15 \% der aktiven Vereinsmitglieder anwesend sind. Bei Beschlussunfähigkeit ist der Vorstand verpflichtet, innerhalb von vier Wochen eine weitere Mitgliederversammlung mit der gleichen Tagesordnung einzuberufen; diese ist ohne Rücksicht auf die Zahl der erschienenen Mitglieder beschlussfähig. Hierauf ist in der Einladung hinzuweisen.
	\item Die Mitgliederversammlung fasst Beschlüsse im Allgemeinen mit einfacher Mehrheit der abgegebenen gültigen Stimmen; Stimmenthaltungen bleiben daher außer Betracht. Zur Änderung der Satzung ist eine Mehrheit von zwei Drittel der abgegebenen gültigen Stimmen, zur Auflösung des Vereins eine solche von drei Vierteln erforderlich. Anträge auf Änderung oder Ergänzung der Satzung müssen spätestens am 7. Tag vor Zusammentritt beim Vorstand eingereicht werden und den
		Stimmberechtigten spätestens am 4. Tag vor Zusammentritt der Versammlung zugänglich sein; die Abstimmung darüber ist nur dann zulässig, wenn der Antrag selbst den Wortlaut der Satzung ausdrücklich ändert oder ergänzt.
	\item Alle Wahlen zu Ämtern und Funktionen, die die Mitgliederversammlung überdauern, erfolgen nach den demokratischen Grundsätzen. Bei der Wahl der Kassen- und Rechnungsprüfer sowie besonderer Beauftragungen jedoch kann von der geheimen Wahl abgesehen werden, wenn sich auf ausdrückliches Befragen kein Widerspruch erhebt.
	\item Gewählt ist, wer die einfache Mehrheit der abgegebenen gültigen Stimmen erhalten hat; bei in sich gleichartigen Ämtern oder Mandaten sind Sammelwahlen zulässig.
	\item Über die Beschlüsse der Mitgliederversammlung ist ein Protokoll aufzunehmen, das von der Versammlungsleitung zu unterzeichnen ist. Es soll folgende Feststellungen enthalten: Ort und Zeit der Versammlung, die Versammlungsleitung, die Zahl der erschienenen Mitglieder, die Tagesordnung, die einzelnen Abstimmungsergebnisse und die Art der Abstimmung. Bei Satzungsänderungen soll der genaue Wortlaut angegeben werden.
	\item Näheres regelt die Geschäftsordnung der Mitgliederversammlung.

\end{enumerate}

\chapter{Finanzprüfer}

\begin{enumerate}
	\item Zur Kontrolle der Haushaltsführung bestellt die Mitgliederversammlung Finanzprüfer. Nach Durchführung ihrer Prüfung setzen sie den Vorstand von ihrem Prüfungsergebnis in Kenntnis und erstatten der Mitgliederversammlung Bericht.
	\item Die Finanzprüfer dürfen nicht dem Vorstand angehören.
\end{enumerate}

\vfill

Beschlossen von der Gründungsversammlung am 18. Mai 2014 (65. Tag des Diskord im YOLD 3180 / Sternzeit -308622,4)\\
Abgeändert durch die Mitgliederversammlung am 09. März 2015 (68. Tag des Chaos im YOLD 3181 / Sternzeit 92789.25)\\
Abgeändert durch die Mitgliederversammlung am 30. April 2016 (47. Tag des Diskord im YOLD 3182)\\

%\input{content/Laborversuch.tex}

% Hier kommmen Tabellen und Bilder
%\listoftables
%\listoffigures

\end{document}
%% Dokument ENDE %%%%%%%%%%%%%%%%%%%%%%%%%%%%%%%%%%%%%%%%%%%%%%%%%%%%%%%%%%

