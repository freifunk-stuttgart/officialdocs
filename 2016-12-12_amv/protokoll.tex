\documentclass[a4paper]{scrartcl}
\usepackage[utf8]{inputenc}
\usepackage{caption}
\usepackage{pdfpages}
\usepackage[ngerman]{babel}
\usepackage{url}
\usepackage{ltxtable}
\usepackage{booktabs}
\usepackage{textcomp}
\parindent0pt
\date{12. Dezember 2016} 

\title{Außerordentliche Mitgliederversammlung des Freifunk Stuttgart e.V.}

\begin{document}

\maketitle

\tableofcontents

\clearpage

\listoftables

\clearpage

\section{Begrüßung und Regularien}
Thomas Renger eröffnet die Versammlung um 19:20 Uhr.

\section{Feststellung ordnungsgemäße Einberufung, Beschlussfähigkeit, TO}
\subsection{Ladung}
Es wird festgestellt, dass ordnungsgemäß geladen wurde. Alle Anwesenden bestätigen, eine fernschriftliche Einladung erhalten zu haben.

\subsection{Feststellung der Beschlussfähigkeit}
\begin{itemize}
\item es sind 14 von 46 Vereinsmitglieder anwesend
\item 15\% der Vereinsmitglieder sind für die Beschlussfähigkeit nötig, die Bedingung ist folglich erfüllt.
\end{itemize}
Es wird vereinbart, dass die Geschäftsordnung laut github-Repository\footnote{\url{https://github.com/freifunk-stuttgart/officialdocs/blob/master/go_gv/go_gv.pdf}} fortgilt. Thomas Renger übernimmt die Versammlungsleitung.

\subsection{Gäste} 
Die Vereinsmitglieder werden befragt, ob Gäste beim Treffen zugelassen sein sollen. Das wird einstimmig bejaht. Außerdem sind zwei Vereinsmitglieder über Mumble fernmündlich zugeschaltet.

\subsection{Tagesordnung}
Um 19:45 kommt ein weiteres Vereinsmitglied hinzu. Es sind jetzt 15 Vereinsmitglieder anwesend.\\
Die Tagesordnung wird einstimmig verabschiedet.\\
\begin{table}[h]
	\begin{tabularx}{\textwidth}{XXX}
		Dafür & Dagegen & Enthaltungen\\
		\toprule
		15 & 0 & 0\\
	\end{tabularx}
	\caption{Tagesordnung}
\end{table}

\section{Gemeinnützigkeit des Vereins}
Um 19:49 kommt ein weiteres Vereinsmitglied hinzu. Es sind jetzt 16 Vereinsmitglieder anwesend.\\
Es wird in offener Abstimmung ein Meinungsbild über den Diskussionsbedarf zu dem Punkt eingeholt.\\
\begin{table}[h]
	\begin{tabularx}{\textwidth}{XXX}
		Dafür & Dagegen & Enthaltungen\\
		\toprule
		3 & 6 & 7\\
	\end{tabularx}
	\caption{Aussprachewunsch zur Gemeinnützigkeit}
\end{table}
Eine weitere Aussprache wird nicht gewünscht, weil die Diskussion bereits in den vorangegangenen Wochen in den Mailinglisten des Vereins geführt worden war. Es folgt daher sofort die Abstimmung.\\
\begin{table}[h]
	\begin{tabularx}{\textwidth}{XXX}
		Dafür & Dagegen & Enthaltungen\\
		\toprule
		2 & 11 & 3\\
	\end{tabularx}
	\caption{Weitere Verfolgung der Gemeinnützigkeit}
\end{table}

\section{Versicherungen}
Christoph Altrock legt die Angebote der Versicherungen dar.
\subsection{Vereinshaftplichtversicherung}
\subsubsection{Erklärung}
Das ist die Grundlage für alle weiteren möglichen Pakete. Die Vereinshaftpflicht versichert alle aktiven Mitglieder in Ihrer Tätigkeit für den Verein.\\
Als Besonderheit wurden hier die sog. Tätigkeitsschäden mit aufgenommen. Das sind die Schäden, die entstehen können, wenn Router und ähnliches auf fremden Grundstücken installiert werden. Auch wurde der Versicherungsschutz dafür erweitert um NICHT-Vereinsmitglieder, die als Helfer mit hinzu gezogen werden.\\
Kosten: ca. 630 EUR p.a.
\subsubsection{Aussprache}
\begin{itemize}
\item Contra
\begin{itemize}
\item teuer
\item brauchen wir das?
\end{itemize}
\item Pro
\begin{itemize}
\item schützt Verein und Vorstand vor Haftung
\end{itemize}
\end{itemize}
\begin{table}[h]
	\begin{tabularx}{\textwidth}{XXX}
		Dafür & Dagegen & Enthaltungen\\
		\toprule
		13 & 1 & 2\\
	\end{tabularx}
	\caption{Vereinshaftpflichtversicherung}
\end{table}
Verein soll in Fragen Vereinshaftpflicht aktiv werden.
\subsection{Veranstalter-Haftpflicht}
\subsubsection{Erklärung}
Für interne, nicht öffentliche Veranstaltungen genügt die Vereinshaftpflicht. Sobald aber auch öffentliche Veranstaltungen ausgerichtet werden oder der Verein an öffentlichen Veranstaltungen teilnimmt, sollte der Haftpflichtschutz um den Baustein Veranstalter-Haftpflicht ergänzt werden.\\
Eine Anmeldung dieser Veranstaltungen muss nicht jedes Mal erfolgen, sondern der Schutz gilt pauschal. Unentgeltliche Helfer, die z.B. beim Auf- und Abbau von Veranstaltungstechnik oder beim Ausschank helfen sind auch automatisch mitversichert.\\
Kosten: ca. 630 EUR p.a.\\
Um 20:37 kommt ein weiteres Vereinsmitglied hinzu. Es sind jetzt 17 Vereinsmitglieder anwesend.\\
\subsubsection{Meinungsbild: Versicherungen weiter diskutieren?}
Es wird beantragt, die weiteren Versicherungsmodelle nicht einzeln in dieser Versammlung durchzusprechen.\\
\begin{table}[h]
	\begin{tabularx}{\textwidth}{XXX}
		Dafür & Dagegen & Enthaltungen\\
		\toprule
		0 & 12 & 5\\
	\end{tabularx}
	\caption{Weitere Versicherungspakete diskutieren}
\end{table}
Die Mehrheit spricht sich gegen eine detaillierte Diskussion weiterer Versicherungen aus.
\clearpage
\section{Event-Nodes}
Philippe Käufer stellt seinen Antrag zum Verleih der Event-Boxen vor.
\subsection{Antrag}
Der Vorstand wird beauftragt ein Team zu benennen, welches sich um die Betreuung, Wartung und Koordination der Event-Boxen kümmert. Das benannte Team wird selbstständig den Verleih an geeignete Interessenten koordinieren.\\ \\
Aufgaben des Teams:
\begin{itemize}
\item Instandhaltung der Event-Boxen
\item Mit potentielle Interessenten Voraussetzungen und Machbarkeit abklären 
\item Die Event-Boxen dürfen extern nur nach Einweisung bzw. unter Betreuung betrieben werden
\item entsprechende Aufwandsentschädigungen festzulegen
\end{itemize}
Vereinsmitglieder sind von dieser Regelung nicht betroffen und können nach vorheriger Einweisung ohne Aufwandsentschädigung auf das Angebot zurückgreifen. Die Event-Boxen werden an Vereinsmitglieder bevorzugt verliehen.\\
\subsection{Abstimmung}
Um 20:48 verlassen zwei Vereinsmitglieder vorübergehend die Sitzung, kommen aber vor der Abstimmung (20:58) wieder hinzu. Zur Abstimmung sind wieder 17 Vereinsmitglieder anwesend.\\
\begin{table}[h]
	\begin{tabularx}{\textwidth}{XXX}
		Dafür & Dagegen & Enthaltungen\\
		\toprule
		16 & 0 & 1\\
	\end{tabularx}
	\caption{Antrag Event-Box-Verleih}
\end{table}
\clearpage
\section{Sonstiges}
Es wird vereinbart, dass der Vorstand in einer Projektgruppe zusammen mit Freiwilligen eine Zukunftsplanung (ohne Gemeinnützigkeit) erarbeiten und bei der nächsten MV vorstellen soll.
\\
\section{Ende}

Die  Sitzung wird um 21:08 geschlossen.

\vfill
\mbox{}\\
Für die Richtigkeit:\\
\\
\\
Stuttgart, 12. Dezember 2016\\
Stuttgart, Sweetmorn, 54. The Aftermath 3182\\
\\
\\
\hfill Versammlungsleiter  \hfill Protokollant \hfill 1. Vorsitzender \hfill

\end{document}