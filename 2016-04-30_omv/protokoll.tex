\documentclass[a4paper]{scrartcl}
\usepackage[utf8]{inputenc}
\usepackage{caption}
\usepackage{pdfpages}
\usepackage[ngerman]{babel}
\usepackage{url}
\usepackage{ltxtable}
\usepackage{booktabs}
\usepackage{textcomp}
\parindent0pt
\date{30. April 2016} 

\title{Ordentliche Mitgliederversammlung des Freifunk Stuttgart e.V.}

\begin{document}

\maketitle

\tableofcontents

\clearpage

\listoftables

\clearpage

\section{Begrüßung}
Thomas Renger eröffnet die Versammlung um  9:59 Uhr.

\section{Feststellung ordnungsgemäße Einberufung, Beschlussfähigkeit, TO}
\subsection{Ladung}
Es wird festgestellt, dass Ordnungsgemäß geladen wurde. Alle Anwesenden bestätigen, eine fernschriftliche Einladung erhalten zu haben.

\subsection{Feststellung der Beschlussfähigkeit}
\begin{itemize}
\item es sind 18 Personen anwesend.
\item es sind 17 Vereinsmitglieder anwesend
\item 15\% der Vereinsmitglieder sind für die Beschlussfähigkeit nötig, die Bedingung ist folglich erfüllt.
\end{itemize}
\subsection{Gäste} 
Die Vereinsmitglieder werden befragt, ob Gäste beim Treffen zugelassen sein sollen. Das wird einstimmig bejaht.

\subsection{Anträge zur Änderung der Tagesordnung}
Es wird beantragt, den TOP 5 (Neuwahlen) der vorläufigen Tagesordnung mit dem TOP 6 (Satzungsänderungen) zu tauschen.
\begin{table}[h]
	\begin{tabularx}{\textwidth}{XXX}
		Dafür & Dagegen & Enthaltungen\\
		\toprule
		13 & 1 & 3\\
	\end{tabularx}
	\caption{Änderung der Tagesordnung}
\end{table}

Die geänderte Tagesordnung wird schließlich einstimmig verabschiedet.

Ferner gilt die Geschäftsordnung laut github-Repository\footnote{\url{https://github.com/freifunk-stuttgart/officialdocs/blob/master/go_gv/go_gv.pdf}} fort.

\section{Wahl der Versammlungsämter}
\begin{itemize}
\item Für das Amt des Versammlungsleiters kandidiert Thomas Renger. Der Vorschlag wird einstimmig angenommen, der Gewählte nimmt das Amt an.
\item Für das Amt des Wahlleiters kandidiert Dominik Lamp. Der Vorschlag wird einstimmig angenommen, der Gewählte nimmt das Amt an.
\item Roland Volkmann wird durch den Wahlleiter als Wahlhelfer bestellt. Er nimmt das Amt an.
\item Für das Amt des Protokollanten kandidiert Hannes Zechmann. Der Vorschlag wird einstimmig angenommen, der Gewählte nimmt das Amt an.
\end{itemize}

\section{Tätigkeitsberichte Vorstand und Kassenprüfer, Entlastung}
\subsection{Vorstandsbericht}
Es berichten Christoph Altrock, Wilhelm Humerez und Thomas Renger.\\
Dinge, die erreicht wurden:
\begin{itemize}
\item Vereinseintragung wurde vorgenommen und ist fertig
\item Das Netz wurde weiter ausgebaut: über 800 Nodes sind online, Clients Spitzenwert: 2800 Nutzer
\item FFS ist Mitglied beim shackspace e.V. geworden
\item FFS hatte Kontakt zum Stadtmarketing Stuttgart
\item Stadtmarketing Esslingen hat uns auch angesprochen
\item Wir helfen in der Flüchtlingshilfe, ca 100 Unterkünfte sind mit FFS versorgt
\item Jugendhäuser wurden angeschlossen
\item Wir waren auf der Hobby und Elektronik-Messe vertreten und werden es das nächste mal wieder anstreben
\item Session auf dem Barcamp Stuttgart
\item Versorgung von TEDx mit FF
\item Veranstaltung "Open" auf der 2015
\item BSZ Leonberg wurde verkabelt
\item Landesakademie ES wurde angeschlossen
\item Palast der Republik
\item Tahiti Bar
\item Tennis Club Herrenberg (sogar als Fördermitglied)
\item Vorträge auf der No-Spy Konferenz gehalten
\item Bäckerei Lang / Cafe Wirth in der Königstraße Stuttgart
\item Landesmuseum Baden-Württemberg hat uns angefragt, FF im Museum auf min. einer Veranstaltung zu machen
\item Altes Schloss Stuttgart hat uns angeboten, die Dachfenster für Distanzfunk nutzen zu dürfen.
\item KKJ Feuerbach wurde angeschlossen
\item Ein Ticketsystem wurde aufgesetzt
\end{itemize}
Dinge, die noch nicht erreicht wurden:\\
Gemeinnützigkeit: Anträge wurden im Herbst 2015 zum Finanzamt geschickt, bisher gibt es keine Rückmeldung. Wird in Abhängigkeit von den späteren Abstimmungsergebnissen weiterhin angestrebt.\\
Ausblick:
\begin{itemize}
\item Flash in den Mai
\item Evangelischer Medientag
\item Hobby \& Elektronik
\item eventuell Bar Camp Stuttgart
\end{itemize}
\subsection{Bericht der Kasse}
Es berichtet Christoph Altrock:\\
\begin{itemize}
\item Einnahmen: 3149 \texteuro
\item Aufwände: 599,53 \texteuro
\item Netto Ertrag: 2549,47 \texteuro
\end{itemize}
\subsection{Kassenprüfung}
Es berichten David Mändlen und Dominik Lamp:\\
Bücher wurden durchgesehen, Bar-Kasse wurde gezählt. Es gibt keine Gründe zur Annahme, dass die Kasse nicht stimmt.\\
Antrag wird gestellt, den Vorstand zu entlasten. Antrag wird einstimmig angenommen.

\section{Satzungsänderungen}
Ein Mitglied verlässt den Raum. Für die folgenden Abstimmungen sind 16 Mitglieder anwesend.

\subsection{Satzungsänderungsantrag 1 (via github issue von "ghost")}
\subsubsection{aktuelle Formulierung}

{[}\dots{]}\\

\subsubsection{ausformulierte geänderte Version}

Änderungsantrag für §9, Vorschlag für neuen Absatz c):\\
\\
c) Außerordentliche Mitgliederversammlungen werden auf Beschluss des Vorstandes abgehalten, wenn die Interessen des Vereins dies erfordern, oder wenn mindestens fünf Mitglieder dies unter Angabe des Zwecks schriftlich oder fernschriftlich beantragen.

\subsubsection{Begründung}

(fehlt)

\subsubsection{Abstimmung}

Es wird beantragt, den Antrag aus formalen Gründen (Begründung fehlt, es ist nicht sicher, ob "ghost" Mitglied ist) abzulehnen. Dieser Antrag wird mit großer Mehrheit angenommen. Somit ist der Satzungsänderungsantrag abgelehnt.

\subsection{Satzungsänderungsantrag 2 (von Albi)}
\subsubsection{aktuelle Formulierung}

{[}\dots{]}\\

\subsubsection{ausformulierte geänderte Version}

{[}\dots{]}\\

\subsubsection{Begründung}

Ich beantrage hiermit den Verein umzubenennen in "Freifunk BW" für Baden Württemberg.
Da der Wirkungskreis weit über die Grenzen von Stuttgart hinausgehen.

\subsubsection{Abstimmung}

(erfüllt nicht die formalen Kriterien an einen Satzungsänderungsantrag)\\

Der Wahlleiter holt sich per Abstimmung ein Stimmungsbild von der Versammlung. Die große Mehrheit ist gegen die Namensänderung.
Es wird festgestellt, dass der Antrag nicht mehrheitsfähig ist.

\begin{table}[h]
	\begin{tabularx}{\textwidth}{XXX}
		Dafür & Dagegen & Enthaltungen\\
		\toprule
		1 & 12 & 3\\
	\end{tabularx}
	\caption{Satzungsänderungsantrag 2}
\end{table}

Es wird gefragt ob die weiteren Satzungsänderungsanträge geheim abgestimmt werden sollen. Beschluss: die Änderungen werden nicht geheim abgestimmt.

\subsection{Satzungsänderungsantrag 3 (von Philippe Käufer)}

§2 Absatz a)\\

Der Absatz soll komplett ersetzt werden durch:
„Zweck des Vereins ist die Förderung der Volksbildung sowie die Verbreitung kabelloser und kabelgebundener Computernetzwerke, die der Allgemeinheit zugänglich sind (freie Netzwerke), insbesondere Freifunk-Netzwerke.“\\

§2 Absatz c)\\
Punkt 3 „Unterstützung des Betriebs freier Netzwerke;“ wird ersatzlos gestrichen.

\subsubsection{aktuelle Formulierung}

Zweck des Vereins ist die Förderung und der Betrieb kabelloser und kabelgebundener Computernetzwerke, die der Allgemeinheit zugänglich sind (freie Netzwerke), insbesondere Freifunk-Netzwerke

\subsubsection{ausformulierte geänderte Version}

Zweck des Vereins ist die Förderung der Volksbildung sowie die Verbreitung kabelloser und kabelgebundener Computernetzwerke, die der Allgemeinheit zugänglich sind (freie Netzwerke), insbesondere Freifunk-Netzwerke.

\subsubsection{Begründung}

Das Bundesfinanzministerium hat am 31.01.2014 ein Schreiben herausgegeben, bei welchem der Anwendungserlass zur Abgabenordnung stellenweise neu ausgelegt wurde.(\url{http://www.bundesfinanzministerium.de/Content/DE/Downloads/BMF_Schreiben/Weitere_Steuerthemen/Abgabenordnung/AO-Anwendungserlass/2014-01-31-Neubekanntmachung-AEAO.pdf?__blob=publicationFile&v=2})\\

Unter anderem betrifft uns Seite 31, Punkt 3.\\
Hier heißt es: „Internetvereine können wegen Förderung der Volksbildung als gemeinnützig anerkannt werden, sofern ihr Zweck nicht der Förderung der (privat betriebenen) Datenkommunikation durch Zurverfügungstellung von Zugängen zu Kommunikationsnetzwerken sowie durch den Aufbau, die Förderung und den Unterhalt entsprechender Netze zur privaten und geschäftlichen Nutzung durch die Mitglieder oder andere Personen dient. [...]“.\\

Das Ziel des Vereins Freifunk Stuttgart e.V. ist nicht das Betreiben von Kommunikationsnetzwerken. Ziel ist vielmehr die Förderung der Volksbildung im Bereich Kommunikationsnetzwerke, das Ideal von freien Netzwerken bekannt zu machen sowie den Open-Source Gedanken zum Wohle der Gesellschaft zu verbreiten.
Dies sollte auch in der Satzung verankert werden.\\

\subsubsection{Abstimmung}

Antrag wurde verlesen. 

\begin{table}[h]
	\begin{tabularx}{\textwidth}{XXX}
		Dafür & Dagegen & Enthaltungen\\
		\toprule
		5 & 6 & 5\\
	\end{tabularx}
	\caption{Satzungsänderungsantrag 3}
\end{table}

Antrag ist abgelehnt.

\subsection{Satzungsänderungsantrag 4 (von Philippe Käufer)}
\subsubsection{aktuelle Formulierung}

Die Versammlung wird mindestens einmal im Kalenderjahr einberufen durch die Ladung aller stimmberechtigten Mitglieder. Die Einladung erfolgt per E-Mail
muss mindestens enthalten:\\
1) den Anlass der Einberufung\\
2) das kalendarische Datum\\
3) den genauen Ort (postalische Adresse)\\
4) die genaue Uhrzeit der Akkreditierung, Beginn und geplantes Ende der Versammlung\\
5) die vorläufige Tagesordnung\\
6) Angaben dazu, wo bereits vorliegende Anträge in Textform aufzufinden und einzusehen sind\\
7) Namen und Amtsbezeichnung des Ladenden.\\

\subsubsection{ausformulierte geänderte Version}

Die Versammlung wird mindestens einmal im Kalenderjahr einberufen durch die Ladung aller stimmberechtigten Mitglieder. Die Einladung erfolgt unter Einhaltung einer Frist von mindestens vier Wochen per E-Mail muss mindestens enthalten:\\
1) den Anlass der Einberufung\\
2) das kalendarische Datum\\
3) den genauen Ort (postalische Adresse)\\
4) die genaue Uhrzeit der Akkreditierung, Beginn und geplantes Ende der Versammlung\\
5) die vorläufige Tagesordnung\\
6) Angaben dazu, wo bereits vorliegende Anträge in Textform aufzufinden und einzusehen sind\\
7) Namen und Amtsbezeichnung des Ladenden.\\

\subsubsection{Begründung}

Die Ladungsfrist der Mitgliederversammlung ist nicht klar geregelt und kann willkürlich erfolgen. Um hier den Mitglieder in Zukunft einen klar definierten Zeitraum für die Vorbereitung und Bedenkzeit über etwaige gestellte Anträge und Tagesordnungspunkte zu ermöglichen, stelle ich den Antrag eine geregelte Ladungsfrist sicher zu stellen.

\subsubsection{Abstimmung}

Antrag wurde verlesen.

\begin{table}[h]
	\begin{tabularx}{\textwidth}{XXX}
		Dafür & Dagegen & Enthaltungen\\
		\toprule
		13 & 1 & 2\\
	\end{tabularx}
	\caption{Satzungsänderungsantrag 4}
\end{table}
Antrag ist angenommen.

\subsection{Satzungsänderungsantrag 5 (von Dominik)}
\subsubsection{aktuelle Formulierung}

§2 Zweck des Vereins; Gemeinnützigkeit\\
{[}\dots{]}\\
b) Der Verein verfolgt ausschließlich und unmittelbar gemeinnützige
Zwecke im Sinne des Abschnitts „Steuerbegünstigte Zwecke“ der Abgabenordnung.

\subsubsection{ausformulierte geänderte Version}

Änderungsantrag
§2 Zweck des Vereins\\
{[}\dots{]}\\
b) - gestrichen -

\subsubsection{Begründung}

Der Verein Freifunk Stuttgart e.V. (FFS) soll in der Lage sein, Gateways zur Aggregation des Internetraffics der einzelnen Nodes bereitzustellen
und zu betreiben.\\
Nach Anwendungserlass des Bundesfinanzministeriums ist dies mit der Gemeinnützigkeit eines Vereins nicht vereinbar. Der Betrieb von Gateways durch FFS ist dennoch wünschenswert, da damit juristische Risiken, die sich aus der möglichen Aufname einer Geschäftstätigkeit als Provider ergeben, vom Verein getragen werden können.

\subsubsection{Abstimmung}

Antrag wurde verlesen.

\begin{table}[h]
	\begin{tabularx}{\textwidth}{XXX}
		Dafür & Dagegen & Enthaltungen\\
		\toprule
		0 & 12 & 4\\
	\end{tabularx}
	\caption{Satzungsänderungsantrag 5}
\end{table}
 Antrag ist abgelehnt
    
\subsection{Satzungsänderungsantrag 6 (von Dominik)}
\subsubsection{aktuelle Formulierung}

§9 Einberufung der Mitgliederversammlung\\
a) Die Versammlung wird mindestens einmal im Kalenderjahr einberufen durch die Ladung aller stimmberechtigten Mitglieder. Die Einladung erfolgt per E-Mail muss mindestens enthalten:\\
1) den Anlass der Einberufung\\
2) das kalendarische Datum\\
3) den genauen Ort (postalische Adresse)\\
4) die genaue Uhrzeit der Akkreditierung, Beginn und geplantes Ende der Versammlung\\
5) die vorläufige Tagesordnung\\
6) Angaben dazu, wo bereits vorliegende Anträge in Textform aufzufinden und einzusehen sind\\
7) Namen und Amtsbezeichnung des Ladenden.\\
b) Ort und Datum der Mitgliederversammlung sollen zudem in den Medien des Vereins bekannt gegeben werden.\\


\subsubsection{ausformulierte geänderte Version}

Absatz a) wird wie folgt abgeändert:\\
Die Versammlung wird mindestens einmal im Kalenderjahr einberufen durch die Ladung aller stimmberechtigten Mitglieder. Die Einladung erfolgt per
E-Mail spätestens sechs Wochen vor dem geplanten Versammlungsdatum und muss mindestens enthalten:

{[}\dots{]}\\

Neuer Absatz c):\\
Ergänzungen zur Tagesordnung sind spätestens zwei Wochen nach Versand der Einladungen beim Vorstand zu beantragen. Eine aktualisierte
Tagesordnung wird spätestens drei Wochen vor der geplanten Mitgliederversammlung analog zur Einladung versandt.

\subsubsection{Begründung}

Es wird explizit eine Frist von sechs Wochen eingeführt, um allen Mitgliedern die Möglichkeit zu geben, Tagesordnungspunkte
einzureichen und allen Mitgliedern die Möglichkeit zu geben, sich auch zu diesen Themen vorab zu informieren und eine Meinung zu bilden.\\

Begründung: Es soll sichergestellt werden, dass die Mitglieder die Möglichkeit haben, sich im Vorfeld der Mitgliederversammlung mit den voraussichtlichen Themen auseinanderzusetzen.\\


\subsubsection{Abstimmung}

Der Antrag wird vom Antragssteller wegen des bereits abgestimmten ähnlichen Antrags zurückgezogen.

\subsection{Satzungsänderungsantrag 7 (von Docloy)}
\subsubsection{aktuelle Formulierung}

§4 Beendigung der Mitgliedschaft:\\
{[}\dots{]}\\
b) Der freiwillige Austritt erfolgt durch gegenüber einem Mitglied des Vorstands in Textform. Der Austritt ist zum Monatsende unter Einhaltung einer Kündigungsfrist von zwei Wochen zulässig.

\subsubsection{ausformulierte geänderte Version}

§4 Beendigung der Mitgliedschaft:\\
{[}\dots{]}\\
b) Der freiwillige Austritt erfolgt gegenüber einem Mitglied des Vorstands in Textform. Der Austritt ist zum Monatsende unter Einhaltung einer Kündigungsfrist von zwei Wochen zulässig.

\subsubsection{Begründung}

Falsche Formulierung (ein „durch“ ist übrig).

\subsubsection{Abstimmung}

In der Aussprache wird festgestellt, dass die neue Formulierung nichts verbessert. Der Antrag wird vom Antragsteller zurückgezogen.

\clearpage
\section{Neuwahlen}

\subsection{Vorsitzender}
Thomas Renger und Christoph Altrock kandidieren zum 1. Vorsitzenden.

\begin{table}[h]
	\begin{tabularx}{\textwidth}{XXXX}
		Renger & Altrock & Enthaltungen & Ungültig\\
		\toprule
		11 & 2 & 1 & 1\\
	\end{tabularx}
	\caption{Geheime Wahl zum 1. Vorsitzenden}
\end{table}
Der Gewählte nimmt das Amt an.

\subsection{Stellvertretende Vorsitzende}
Adrian Reyer, Wilhelm Humerez und Philippe Käufer kandidieren zu stellvertretenden Vorsitzenden.

\begin{table}[h]
	\begin{tabularx}{\textwidth}{XXX}
		Humerez & Käufer & Reyer\\
		\toprule
		17 & 17 & 16 \\
	\end{tabularx}
	\caption{Geheime Zustimmungswahl der stellvertretenden Vorsitzenden}
\end{table}
Alle Kandidaten sind gewählt. Die Gewählten nehmen die Ämter an.

\subsection{Schatzmeister}
Christoph Altrock kandidiert zum Schatzmeister.

\begin{table}[h]
	\begin{tabularx}{\textwidth}{XXX}
		Dafür & Dagegen & Enthaltungen\\
		\toprule
		15 & 1 & 1\\
	\end{tabularx}
	\caption{Geheime Wahl zum Schatzmeister}
\end{table}
Der Gewählte nimmt das Amt an.

\subsection{Kassenprüfer}
Kandidaten für die Kassenprüfer: Hannes Zechmann, David Mändlen, Roland Volkmann\\
Die Kandidaten werden einstimmig per Akklamationgewählt\\
Die Gewählten nehmen die Ämter an.

\clearpage

\section{Verschiedenes }

Es wird abgestimmt, die Auswahl einer Veranstaltungsversicherungen auf die Vorstandssitzung zu vertagen.\\

Die  Sitzung wird um 12:05 geschlossen.

\vfill
\mbox{}\\
Für die Richtigkeit:\\
\\
\\
Stuttgart, \today\\
Stuttgart, Prickle-Prickle, 18. Confusion 3182\\
\\
\\
\hfill Versammlungsleiter \hfill Wahlleiter \hfill Protokollant \hfill 1. Vorsitzender \hfill

\end{document}