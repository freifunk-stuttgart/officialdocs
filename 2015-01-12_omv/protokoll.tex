\documentclass[a4paper]{scrartcl}
\usepackage[utf8]{inputenc}
\usepackage{caption}
\usepackage{pdfpages}
\usepackage[ngerman]{babel}
\usepackage{url}
\usepackage{ltxtable}
\usepackage{booktabs}

\title{Ordentliche Mitgliederversammlung des Freifunk Stuttgart e.V.}

\begin{document}

\maketitle

\tableofcontents

\clearpage

\listoftables

\clearpage

\section{Begrüßung}
Martin Eitzenberger eröffnet die Versammlung um 19:33 Uhr.

Michael Schommer begrüßt die Anwesenden.

\section{Feststellung ordnungsgemäße Einberufung, Beschlussfähigkeit, TO}

Es wurde festgestellt, dass die Einladung fristgemäß zugegangen ist, die Versammlung beschlussfähig ist und die in der Einladung zugegangene Tagesordnung gültig ist.
Ferner gilt die GO laut github-Repository\footnote{\url{https://github.com/freifunk-stuttgart/officialdocs/blob/master/go_gv/go_gv.pdf}} fort.\\
\\
\\
Die Versammlungs- und Wahlleitung übernimmt Martin Eitzenberger, das Protokoll David Mändlen.\\
\\
Alle 3 Versammlungsämter werden per Akklamation mit großer Mehrheit bestätigt. Die Gewählten nehmen die Ämter an.

\clearpage

\section{Tätigkeitsberichte Vorstand, Kassenprüfer}
\subsection{1. Vorsitzender}
Martin Eitzenberger trägt seinen Tätigkeitsbericht vor.
Tätigkeitsbericht liegt zum Zeitpunkt der Protokollerstellung nicht schriftlich vor.

\subsection{2. Vorsitzender}
Michael Schommer trägt seinen Tätigkeitsbericht vor.
Tätigkeitsbericht liegt zum Zeitpunkt der Protokollerstellung nicht schriftlich vor.

\subsection{Schatzmeister}
David Mändlen trägt seinen Tätigkeitsbericht vor.
Tätigkeitsbericht liegt als Anhang zum Protokoll vor.

\subsection{Kassenprüfer}
Philippe Käufer trägt den Bericht der Kassenprüfer vor.
Prüfbericht liegt als Anhang zum Protokoll vor.

\subsection{Entlastung Vorstand}
\begin{table}[h]
	\begin{tabularx}{\textwidth}{XXX}
		Dafür & Dagegen & Enthaltungen\\
		\toprule
		16 & 0 & 2\\
	\end{tabularx}
	\caption{Entlastung des Vorstands}
\end{table}

\clearpage

\section{Anträge}
\begin{table}[h]
	\begin{tabularx}{\textwidth}{XXX}
		Dafür & Dagegen & Enthaltungen\\
		\toprule
		17 & 0 & 1\\
	\end{tabularx}
	\caption{Antrag gemäß Anhang}
\end{table}


Antrag laut Einladung

\clearpage

\section{Neuwahlen}
\subsection{Vorsitzender}
Thomas Renger kandidiert zum 1. Vorsitzenden.
Nach der Kandidatenvorstellung folgt auf die Frage nach Fragen an den Kandidaten Grillenzirpen.

\begin{table}[h]
	\begin{tabularx}{\textwidth}{XXX}
		Dafür & Dagegen & Enthaltungen\\
		\toprule
		16 & 0 & 2\\
	\end{tabularx}
	\caption{Akklamationswahl zum 1. Vorsitzenden}
\end{table}
Der Gewählte nimmt das Amt an.

\subsection{Schatzmeister}
Christoph Altrock kandidiert zum Schatzmeister.
Nach der Kandidatenvorstellung folgt auf die Frage nach Fragen an den Kandidaten Grillenzirpen.

\begin{table}[h]
	\begin{tabularx}{\textwidth}{XXX}
		Dafür & Dagegen & Enthaltungen\\
		\toprule
		17 & 0 & 1\\
	\end{tabularx}
	\caption{Akklamationswahl zum Schatzmeister}
\end{table}
Der Gewählte nimmt das Amt an.

\subsection{Stellvertretende Vorsitzende}
Adrian Reyer kandidiert zum 1. stellvertretenden Vorsitzenden.

\begin{table}[h]
	\begin{tabularx}{\textwidth}{XXX}
		Dafür & Dagegen & Enthaltungen\\
		\toprule
		17 & 0 & 1\\
	\end{tabularx}
	\caption{Akklamationswahl zum 1. stellvertretenden Vorsitzenden (A. Reyer)}
\end{table}
Der Gewählte nimmt das Amt an.

\clearpage

Michael Schommer und Wilhelm Humerez kandidieren als weitere stellvertretende Vorsitzende.

\begin{table}[h]
	\begin{tabularx}{\textwidth}{XXX}
		Dafür & Dagegen & Enthaltungen\\
		\toprule
		17 & 0 & 1\\
	\end{tabularx}
	\caption{Akklamationswahl zum 2. stellvertretenden Vorsitzenden (M. Schommer)}
\end{table}
Der Gewählte nimmt das Amt an.

\begin{table}[h]
	\begin{tabularx}{\textwidth}{XXX}
		Dafür & Dagegen & Enthaltungen\\
		\toprule
		17 & 0 & 1\\
	\end{tabularx}
	\caption{Akklamationswahl zum 3. stellvertretenden Vorsitzenden (W. Humerez)}
\end{table}
Der Gewählte nimmt das Amt an.

\subsection{Kassenprüfer}
Dominik Lamp und David Mändlen kandidieren als Kassenprüfer.

\begin{table}[h]
	\begin{tabularx}{\textwidth}{XXX}
		Dafür & Dagegen & Enthaltungen\\
		\toprule
		17 & 0 & 1\\
	\end{tabularx}
	\caption{Blockabstimmung über die Kassenprüfer}
\end{table}
Der Gewählte nimmt das Amt an.

\clearpage

\section{Verschiedenes }
Entfällt.

\vfill
\mbox{}\\
Für die Richtigkeit:\\
\\
\\
Stuttgart, \today\\
Stuttgart, Boomtime, 12. Chaos 3181\\
Stuttgart, Sternzeit 73064,98\\
\\
\\
\hfill Versammlungsleiter \hfill Wahlleiter \hfill Protokollant \hfill 1. Vorsitzender \hfill

\end{document}
